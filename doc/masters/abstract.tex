\begin{abstract}

Application performance monitoring is really challenging domain. It is because of multidimensional nature of this problem. Such a system must leave as small resources footprint as possible to not pollute measured results, it must give the user deep insight into the application but should not change application\rq{}s life cycle too much. Additionally the problem gets more and more difficult with increasing scale of applications in subject. Mostly due to those factors, author of this thesis decided to take a closer look into the problem and propose own solution, by implementing new system called SemSimMon - Semantic Simple Monitor. Solution that will be simple to use yet powerful. Build on semantic approach, which makes working with application easier and also simplifies extending of the application.

In order to create good software, one must first perform deep analysis of problem domain. Such a analysis was performed, existing distributed application measuring systems and related technologies were taken into account. Projects like Knowledge-based Workflow System for Grid Applications (K-WfGrid), On-Line Monitoring Interface Specification (OMIS), Grid Monitoring Architecture (GMA), perfSonar, SCALEA, ,  DIstributed Performance Analysis Service (DIPAS), and related were covered. Additionally to properly employ semantics in application, deep analysis of Semantic Web framework was also performed. During this analysis, author described most important projects of the framework, namely Resource Description Framework (RDF), and Web Ontology Language (OWL).

As a first step in actual work on SemSimMon, requirements were gathered, both functional and non-functional.  and structured into strict use cases. Additionally functional requirements were structured into 24 strictly defined use cases, covering application functionality. Application allows user to add JMX or OCM-G based resources, manage them, create and manage measurements, as well as dynamically change system ontology thus changing the structure of resources tree. Additionally user may create multiple visualizations, attach measurements and view results updated on-line. In next steps, system architecture and implementation details were provider. Application was decomposed into 6 high level components: GUI, Monitoring Hub, Monitoring Hub Application, Knowledge, JMX Transport Proxy and OCM-G Transport Proxy. Additionally further decomposition of those high level components were also described, which creates top to bottom architectural analysis. System was implemented using Java programming language using modern technologies like Apache Pivot, Spring, RMI. 

In order to proof SemSimMon usability several tests were performed. As first, performance test were conducted. Application was attached to three instances of test application that runs processing emulations, then several metrics and visualization were created to see how system behaves while working on load. For profiling, JProfiler was used. Analysis shown that there aren\rq{}t performance caveats. At the end, usability test were performed. SemSimMon were used to monitor test application - linear equations solver with J2EE web interface, and MPI-based solving module. This test proved that application works with real use cases, as it allowed to find bug in solver and showed that approach to implementation of test application was appropriate.



\end{abstract}

