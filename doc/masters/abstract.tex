\begin{abstract}

Application performance monitoring is a challenging domain, mostly because of the multidimensional nature of this problem. Such a system must do as little resources footprint as possible, not to contaminate measured results, it must provide the user deep insight into the application but should not modify a program's development cycle too much. Additionally the problem gets more and more complex with the increasing scale of applications in subject. Mostly due to these factors, the author of this thesis decided to take a closer look into the problem and introduce alternative solution, called SemSimMon - Semantic Simple Monitor. The solution proposed will be easy to use yet powerful and built using semantic-based approach, which will make working with application easier and which also simplifies extending of the application.

In order to make high-quality software, one must first perform a deep analysis of the problem domain. Such a study has been carried out, the existing distributed application measuring systems and related technologies were taken into account. The projects like Knowledge-based Workflow System for Grid Applications (K-WfGrid), On-Line Monitoring Interface Specification (OMIS), Grid Monitoring Architecture (GMA), perfSonar, SCALEA, DIstributed Performance Analysis Service (DIPAS), and related were analyzed. Additionally, to use semantics in the application properly, a deep analysis of the Semantic Web framework was also performed. During this analysis, the author described the most noteworthy projects of the framework, namely Resource Description Framework (RDF), and Web Ontology Language (OWL).

As a first step in the actual work on SemSimMon, a catalogue of requirements was gathered, both functional and nonfunctional. Additionally, the functional requirements were structured into 24 strictly defined use cases, covering the application functionality. The application allows the user to add JMX or OCM-G based resources, manage them, create and control measurements, as well as to dynamically change the system ontology; thus, changing the structure of a resources tree. Additionally the user may create many types of visualizations, attach measurements and view results, updated on-line. In next steps, the system design and implementation details were provided. The application was divided into 6 high level components: GUI, Monitoring Hub, Monitoring Hub Application, Knowledge, JMX Transport Proxy and OCM-G Transport Proxy. Additionally further decomposition of these high-level components was also performed, resulting in a top to bottom architectural review. The system was implemented using the Java programming language using modern technologies like Apache Pivot, Spring, RMI.

Several usability tests were performed, in order to prove SemSimMon\rq{}s usability. As first, performance evaluation was conducted. The application was attached to three instances of the sample application. In the next step, several metrics and visualizations were created to examine how the system behaves while working under significant load. For profiling purposes, JProfiler was used. A subsequent analysis shows that there are no performance caveats when using the program. At the end, usability test was performed. The SemSimMon was used to monitor another sample service - a linear equations solver with a J2EE web interface, and a solving module that was implemented using the MPI library. This test proved that the application can handle the real world example. 

\end{abstract}

