\begin{abstract}

Application performance monitoring is a challenging domain, mostly because of a multidimensional nature of this problem. Such a system must do as little resources footprint as possible, not to contaminate measured results, it must provide the user deep insight into the application but should not modify a program's development cycle too much. Additionally the problem gets more and more complex with increasing scale of applications in subject. Mostly due to those factors, author of this thesis decided to take a closer look into the problem and introduce alternative solution, called SemSimMon - Semantic Simple Monitor. Proposed solution will be easy to use yet powerful and built using semantic approach, which will make working with application easier and which also simplifies extending of the application.

In order to make high-quality software, one must first perform deep analysis of the problem domain. Such a study was performed, existing distributed application measuring systems and related technologies were taken into account. Projects like Knowledge-based Workflow System for Grid Applications (K-WfGrid), On-Line Monitoring Interface Specification (OMIS), Grid Monitoring Architecture (GMA), perfSonar, SCALEA, DIstributed Performance Analysis Service (DIPAS), and related were covered. Additionally to use semantics in application properly, deep analysis of Semantic Web framework was also performed. During this analysis, author described most noteworthy projects of the framework, namely Resource Description Framework (RDF), and Web Ontology Language (OWL).

As a first step in actual work on SemSimMon, requirements were gathered, both functional and nonfunctional. Additionally functional requirements were structured into 24 strictly defined use cases, covering application functionality. Application allows the user to add JMX or OCM-G based resources, manage them, create and control measurements, as well as dynamically change system ontology thus changing the structure of resources tree. Additionally user may create multiple visualizations, attach measurements and view results updated on-line. In next steps, system design and implementation details were provided. Application was divided into 6 high level components: GUI, Monitoring Hub, Monitoring Hub Application, Knowledge, JMX Transport Proxy and OCM-G Transport Proxy. Additionally further decomposition of those high-level components was also performed, resulting in top to bottom architectural review. System was implemented using Java programming language using modern technologies like Apache Pivot, Spring, RMI.

Several usability tests were performed, in order to prove SemSimMon usability. As first, performance evaluation was conducted. Application was attached to three instances of the sample application. In the next step, several metrics and visualization were created to examine how system behaves while working on significant load. For profiling, JProfiler was used. Analysis has shown that there are no performance caveats when using the program. At the end, usability test were performed. SemSimMon was used to monitor another sample service - linear equations solver with J2EE web interface, and a solving module that was implemented using MPI library. This test proved that the application works well in the real world example. It allowed to find a bug in solver and showed that design of test application was appropriate.



\end{abstract}

