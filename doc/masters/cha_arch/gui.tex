 
%---------------------------------------------------------------------------
% Gui component.
%
%---------------------------------------------------------------------------


\section{GUI component}
\label{sec:arch_gui}

Graphical user interfaces are most commonly used form of interaction with user in modern computing. Designing decent
user interface is especially important due to aim of this work - creation of measurement and visualization tool. Although GUI component has rather limited role of giving user control over application and allow to view results of the work it's in fact most complex component. Additionally any shortcoming in GUI is relatively difficult to be hidden by user, and that makes UI/UX (User Interface/User Experience) engineering so important.

While designing user interface, I've been trying to follow few general principles. First of all, I wanted resulting
interface to be as transparent as possible. Users should focus on their tasks instead of learning how to use the tool. Because of that, application shouldn't be bloated with unnecessary options and steps that needs to be performed by users need to achieve their goals should be as short as possible. Next equally important interface item are visualizations. Because visualizations are most important functionality of application, they need special concern. User should be able to view charts with results without any interruptions or any other UI components that might be disturbing. What is also important, GUI must be coherent to free user from chaos of being spread across multiple windows, desktops. 

From business logic point of view, GUI component will extensively use MVC design pattern\cite{gamma1995}. Each form,
window or more complex section needs to have it's own View, responsible for presentation, Controller that collects
user's events and updates view on user's actions or system events. Application will have shared model, which will be
access point for underlaying, external to GUI components.
 

\subsection{Interface Mockups}

In this section, I will try to describe mockups of most important views. First one, depicted in Figure~\ref{fig:mock_main} shows main application view. Left edge of application window contains vertical tab pane controller that will allow user to easily switch views covering most important application contexts - resources, measurements and visualizations. Another benefit of such a approach is that application has a lot of space for what user actually needs. The second most important component of main view is menu bar placed on top. It allows user to perform bulk operations (e.g. pause all measurements) regardless of view he or she is currently. 

All subsequent mockups covers context specific views that will be rendered in main view's central pane.

\begin{figure}[ht]
  \centering
  \includegraphics[width=0.7\textwidth]{mock_main}
  \caption{GUI application main view mockup}
  \label{fig:mock_main}
\end{figure}

When user starts to work with application He or She first must add resources that will be measured. That's why resources view is first, initial section displayed directly after startup. Resources Mockup can bee found in Figure~\ref{fig:mock_resources}. The view is divided into two high level logical parts - the left pane covers global
resources context - users can browse measurement tree as well as add new resources into it. The right pane is specific
to resource selected by user from tree on left and contains accordion with two sections. The upper one shows resource's static
attributes. The one below allows user to see snapshot of resource's dynamic state and check all of it's capabilities at
given point of time. To refresh those values, Refresh button has been provided. Additionally, below accordion there are
several buttons allowing performing actions on given, selected resource.

\begin{figure}[ht]
  \centering
  \includegraphics[width=0.7\textwidth]{mock_resources}
  \caption{GUI application resources pane}
  \label{fig:mock_resources}
\end{figure}

Next figure, Figure~\ref{fig:mock_measurements} covers measurements context of application. All measurements created
are listed on left side of this view. After selecting one accordion on right side shows details of selected
measurement. The upper section contain table with general information. The lower one shows all values gathered since creation of selected resource. Using controls under measurements list, user can
pause, resume or remove measurement. Additionally in this section user may copy the results to clipboard, in CSV format which can be then easily imported to any spreadsheet application for further analysis. 

\begin{figure}[ht]
  \centering
  \includegraphics[width=0.7\textwidth]{mock_measurements}
  \caption{GUI application measurements pane}
  \label{fig:mock_measurements}
\end{figure}



\begin{figure}[ht]
  \centering
  \includegraphics[width=0.7\textwidth]{mock_vis_clean}
  \caption{GUI application visualizations pane, clean view}
  \label{fig:mock_vis_clean}
\end{figure}

Appropriate visualizations display is probably most difficult to design from user experience engineering perspective. During designing this UI component I used modest web browsers interfaces as an inspiration. The effects can be seen in Figure~\ref{fig:mock_vis_clean} - in proposed solution every visualization is being rendered on separate, horizontal tab. To add new visualization, user should just click last tab with prompting icon. To remove visualization - user simply clicks cross icon on visualization tab - just as He or She would close tab in browser. This gives visualization chart as much space as
application can give, and still allows user to easily control creation and disposal of visualization.

The biggest problem with such an approach is where to place controls of visualization. User must be able to choose which
measurements should be included in given visualization, as well as type of chart or define other configuration settings. To address this need, management pane was designed. It is hidden by default and is being displayed to the user, on mouse over right edge of chart, marked with '<' signs. Layout of this pane is depicted in Figure~\ref{fig:mock_vis_options}. The
settings pane contains form, divided into severals sections:

\begin{itemize}
 \item Visualization Options - user can here configure label of whole visualization, as can be seen on tab pane
 \item Chart Options - allows setting chart's title (rendered on chart's graphics) and choose type of the chart. User will
be able to use line (XY scatter), pie, bar and spider web chart types.
 \item Measurements - gives control over which measurements should be included in given visualization. To add
measurement into visualization, user should click Add button and from displayed dialog choose which measurements should
be included. System doesn't give any constraints on measurements to be chosen, so it's up to the user to prepare
meaningful visualization. User will be able to remove given measurement from chart, by simply selecting it from list
and clicking Remove button.
 \item Actions - allows user to pause, resume visualization. Additionally user can copy to clipboard image containing
snapshot of visualization's chart.
\end{itemize}


\begin{figure}[ht]
  \centering
  \includegraphics[width=0.7\textwidth]{mock_vis_options}
  \caption{GUI application visualizations pane, view with options pane}
  \label{fig:mock_vis_options}
\end{figure}


