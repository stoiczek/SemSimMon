%---------------------------------------------------------------------------
% Content of Chapter 5 - System architecture
%
%---------------------------------------------------------------------------


\chapter{System architecture}
\label{cha:sys_arch}


\section{Introduction}
\label{sec:ch5_gui}

Design stage is probably most important in whole application life cycle. Decisions made during this stage affects 
all following stages. Properly designed architecture of system or application reduces possibility of pitfalls during
implementation stage. It also allows easier extension, growth of application, as well as better integration with other
systems, thus makes application lifetime longer. 

While designing application I have been trying to follow few well known, software design principles, like:
\begin{itemize}
 \item {\bf Single Responsibility
principle}~~~~~~~~~~~~~~~~~~~~~~~~~~~~~~~~~~~~~~~~~~~~~~~~~~~~~~~~\linebreak
Application should be divided into distinct features with as
little overlap as possible. Architecture should strain to minimize amount of interaction points to achieve high cohesion
and low coupling.
 \item {\bf Principle of Least Knowledge (Law of Demeter~-~LoD)}~~~~~~~~~~~~~~~~~~~~~~~~~~~~~~~~~~~~~~~~~\linebreak
Each component or module should be responsible for only a specific feature or functionality, or aggregation of cohesive
functionality.
 \item {\bf Don't repeat yourself (DRY)}~~~~~~~~~~~~~~~~~~~~~~~~~~~~~~~~~~~~~~~~~~~~~~~~~~~~~~~~\linebreak
You should only need to specify intent in one place. For example, in terms of application design, specific functionality
should be implemented in only one component; the functionality should not be duplicated in any other component.
 \item {\bf Keep It Simple Stupid! (KISS)}~~~~~~~~~~~~~~~~~~~~~~~~~~~~~~~~~~~~~~~~~~~~~~~~~~~~~~~~\linebreak
It's a broad principle, but works in software architecture domain quite well. Generally speaking, it states that while
designing application, one should try to avoid creating unnecessarily complex structures.
\end{itemize}

To be able to meet both functional and nonfunctional requirements, 


In this chapter, I would like to describe architecture of proposed system. First I would like to distinguish high level
components, describe them roughly and present most important data flows between components.
(Section~\ref{sec:ch5_decomposition}). Then, I will try to analyze each component separately, constituting good base
for further development
(Sections~\ref{sec:ch5_gui},~\ref{sec:ch5_monitoring_hub},~\ref{sec:ch5_knowledge},~\ref{sec:ch5_tproxy})

%---------------------------------------------------------------------------


\input{chapter5/ch5_decomposition}
\input{chapter5/ch5_monitoring_hub}
\input{chapter5/ch5_monitoring_hub_application}
\input{chapter5/ch5_knowledge}
\input{chapter5/ch5_tproxy}
\input{chapter5/ch5_gui}



