%---------------------------------------------------------------------------
% Content of Chapter 5 - System architecture
%
%---------------------------------------------------------------------------


\chapter{System architecture}
\label{cha:sys_arch}

\parbox{0.8\textwidth}{
\makebox{\large Abstract}

{\small
This chapter focuses on a design aspect of software development. At the beginning, I would like to share basic principles, that I have been trying to follow while designing the application. Rest of the chapter covers top-to-bottom analysis of system architecture. First there is a decomposition of top-level components and their interactions (modules). Following subsequent sections cover more detailed analysis and decomposition of high-level modules.
}
}

\section{Introduction}
\label{sec:gui}

Design stage is probably most influential in whole life cycle of an application. Decisions made during this stage affect all following ones. Properly designed architecture of the system or application reduces the possibility of pitfalls during the implementation stage. It also allows easier extension, growth of application, as well as better integration with other systems, thus extends application lifetime.

While designing application, I have been trying to follow few, well known, software design principles, like:

\begin{itemize}

\item {\bf Single Responsibility

principle}~~~~~~~~~~~~~~~~~~~~~~~~~~~~~~~~~~~~~~~~~~~~~~~~~~~~~~~~\linebreak Application should be divided into distinct features with as little overlap as possible. Architecture should strain to minimize the amount of interaction points to achieve a high cohesion and low coupling.

\item {\bf Principle of Least Knowledge (Law of Demeter~-~LoD)}~~~~~~~~~~~~~~~~~~~~~~~~~~~~~~~~~~~~~~~~~\linebreak Each component or module should be responsible for only a specific feature or functionality, or aggregation of cohesive functionality.

\item {\bf Do not repeat yourself (DRY)}~~~~~~~~~~~~~~~~~~~~~~~~~~~~~~~~~~~~~~~~~~~~~~~~~~~~~~~~\linebreak One should only need to specify intent in one place. For example, in terms of application design, functionality should be implemented in only one component; the functionality should not be duplicated in any other component.

\item {\bf Keep It Simple Stupid! (KISS)}~~~~~~~~~~~~~~~~~~~~~~~~~~~~~~~~~~~~~~~~~~~~~~~~~~~~~~~~\linebreak it is a broad principle, but works in software architecture domain quite well. Generally speaking, it states that while designing application, one should try to avoid creating unnecessarily complex structures.
\end{itemize}

In this chapter, I would like to describe the architecture of the proposed system. First, in Section~\ref{sec:arch_decomposition} I will try to distinguish high-level components, describe them roughly and present most critical data flows between them. Following subsequent sections (\ref{sec:arch_gui}, \ref{sec:arch_monitoring_hub}, \ref{sec:arch_monitoring_hub_application} , \ref{sec:arch_knowledge}, and \ref{sec:arch_tproxy}) contain deeper analysis of each high-level component.


%---------------------------------------------------------------------------


%---------------------------------------------------------------------------
% System Decomposition.
%
%---------------------------------------------------------------------------
\section{Decomposition}
\label{sec:arch_decomposition}

The high level system architecture is depicted in Figure~\ref{fig:arch_overall}. 

\begin{figure}[ht]
\centering
\includegraphics[width=1\textwidth]{arch_flows}
\caption{Overall system architecture}
\label{fig:arch_overall}
\end{figure}

At the design stage, several high level components have been introduced. All distinguished components, with relationships between them can be found in Figure~\ref{fig:decomposition_overall}. Each of these components should be interpreted as an independent one, which means that it is either a standalone application or a library, with API containing one or more interfaces. This API is then shared between the provider, which is a component that realizes the given interface and the consumer, a module that uses those functionalities to realize its own aims. The next subsection tries to list those components and describe them roughly.

\begin{figure}[ht]
\centering
\includegraphics[width=1\textwidth]{arch_overall}
\caption{Overall system decomposition}
\label{fig:decomposition_overall}
\end{figure}

\subsection{Components overview}

As can be seen in Figure~\ref{fig:arch_overall}, the system has been decomposed into 8 high-level components. They are:

\begin{itemize}

\item {\bf GUI}~~~~~~~~~~~~~~~~~~~~~~~~~~~~~~~~~~~~~~~~~~~~~~~~~~~~~~~~\linebreak
Graphic User Interface is a standalone, desktop application, used directly by the user. It provides facilities that allow the management of the whole system. It does not perform any measurements or analysis. It only enables control over other components (direct or indirect) and visualizes the results of measurements.
The GUI application contains an embedded Monitoring Hub component, which allows the system to operate also in the smallest scale - a single measuring process with one or more measured processes attached directly. Additionally, it connects to the Monitoring Hub Application component, which allows using a remote Monitoring Hub.

\item {\bf Monitoring Hub}~~~~~~~~~~~~~~~~~~~~~~~~~~~~~~~~~~~~~~~~~~~~~~~~~~~~~~~~\linebreak
Monitoring hub is the most crucial component. Contains all the logic needed for a resource and a measurement management. It uses the Knowledge and Transport Proxies (one or more implementations) components to fulfill its duties. Only the JMX and the OCM-G transport proxies will be provided at this stage of a development, but the proposed architecture allows an easy addition of new proxies.
The Monitoring Hub does not work as a standalone application. Instead, it is in the form of a library (Java JAR) that other components will use. Additionally, the Monitoring Hub component is used by the GUI (in embedded mode) and the Monitoring Hub Application.

\item {\bf Monitoring Hub Application}~~~~~~~~~~~~~~~~~~~~~~~~~~~~~~~~~~~~~~~~~~~~~~~~~~~~~~~~\linebreak
Monitoring Hub Application will be a standalone, command line application (or a daemon if possible) that exposes services of the Monitoring Hub to other components with a remote access through the network (either LAN or WAN). It can accept remote connections from GUI components.
Its main responsibility is to allow the system to work in a distributed manner. Having a process that is separate, and independent from the GUI application, which continuously measures a work of the long-lasting jobs is crucial to allow the system to scale up.


\item {\bf Knowledge}~~~~~~~~~~~~~~~~~~~~~~~~~~~~~~~~~~~~~~~~~~~~~~~~~~~~~~~~\linebreak
The knowledge component realizes the semantics-based approach of the application. It provides ontology functionalities to the Monitoring Hub. It is responsible for initializing the ontology database and responses to all queries issued by the Monitoring Hub. These queries might be related to the relationships between resource types, resources or capabilities. This component is in a form of a library that is a dependency for the Monitoring Hub, and must be included in both the GUI and the Monitoring Hub Application distributions.

\item {\bf JMX Transport Proxy}~~~~~~~~~~~~~~~~~~~~~~~~~~~~~~~~~~~~~~~~~~~~~~~~~~~~~~~~\linebreak
A transport proxy component, which can communicate with JVM processes using the JMX protocol. Its main responsibilities include: establishing a connection with a JVM, mapping between a generic, knowledge-based resource or a capability value request into a Java component or a JMX query.

\item {\bf OCM-G Transport Proxy}~~~~~~~~~~~~~~~~~~~~~~~~~~~~~~~~~~~~~~~~~~~~~~~~~~~~~~~~\linebreak
A transport proxy component that communicates with an OCM-G MainSM monitor. Its responsibilities are similar to those of the JMX Transport Proxy: establish and maintain a connection to the MainSM, translate queries given in ontology terms to OMIS requests.

\end{itemize}

\subsection{Interfaces overview}

To decouple all the proposed components, the system will be using the following interfaces:

\begin{itemize}

\item {\bf Monitoring Hub API}~~~~~~~~~~~~~~~~~~~~~~~~~~~~~~~~~~~~~~~~~~~~~~~~~~~~~~~~\linebreak
It is an interface for the core system's logic. It describes methods that allow for management of every aspect of the system:  measurements, visualizations and resources. It is realized by the Monitoring Hub component and used by the GUI. 

\item {\bf Monitoring Hub Remote API}~~~~~~~~~~~~~~~~~~~~~~~~~~~~~~~~~~~~~~~~~~~~~~~~~~~~~~~~\linebreak
This interface is a derivative of the Monitoring Hub API. It contains the same set of functionalities, with addition of the operations specific to the remote access. This includes: registration of a remote listener, a remote interface wrapper for the Monitoring Hub allowing passing remote exceptions.

\item {\bf Transport API}~~~~~~~~~~~~~~~~~~~~~~~~~~~~~~~~~~~~~~~~~~~~~~~~~~~~~~~~\linebreak
The Transport API is a common interface for a communication with data access services. Currently it is implemented by the JMX and the OCM-G Transport Proxy components. To add support for other data sources in the future, it is necessary to implement this interface.

\item {\bf Knowledge API} ~~~~~~~~~~~~~~~~~~~~~~~~~~~~~~~~~~~~~~~~~~~~~~~~~~~~~~~~\linebreak
Interface describing operations related to the ontology maintenance and usage. This interface is realized by the Knowledge component and used by Monitoring Hub. 

\end{itemize}

\subsection{Most important data flows}

Figure~\ref{fig:comm_add_resource} contains the communication diagram covering the add a new resource action. This action is initialized by the user actor. User starts a flow by performing an action (like button click or wizard - not covered here) on the GUI component. In a response to this event, the GUI sends a request to the Monitoring Hub, asking to register the new resource, using parameters provided by the user. In the subsequent step, the Monitoring Hub first looks up in its dictionary of all registered transport proxies, for the one that can communicate with a resource specified by the user. After finding a valid transport proxy, it passes the registration request to it. In this case, the JMX Transport Proxy tries to initialize a connection to the JVM using a URL provided by the user in the first step. After a successful connection, the JmxTransportProxy creates and initializes a Resource object. This process includes gathering basic attributes, describing the object as well as setting all meta data that the proxy will need to operate with a given resource, in the future. After finishing this step, a fully initialized resource is being returned to the Monitoring Hub. Knowing that the transport proxy has been properly attached to the newly registered resource, the Monitoring Hub notifies about a new resource all listeners, in this case the GUI component. After issuing this notification, it will try to discover all children of this resource. To do it, first it must obtain URLs of child types, so it sends a request to the Knowledge component. Having the types of the child resources, the Monitoring Hub requests the transport proxy to discover the children of an already registered resource and of the given types. Again, in this case the JMX transport proxy will translate a discovery request into JMX queries, to discover the child resources. Discovered children are then returned to the Monitoring Hub as resource objects. All these objects are registered by the Monitoring Hub and a notification is being issued to the listeners.

\begin{figure}[ht]
\centering
\includegraphics[width=0.9\textwidth]{comm_add_resource}
\caption{Communication diagram - adding of new resource}
\label{fig:comm_add_resource}
\end{figure}

The addition of a new measurement is a bit simpler than the registration of new resources. The message exchange needed to achieve it can be found in Figure~\ref{fig:comm_add_measurement}. In this case, similarly as above, the action is initialized by user. He or she chooses a resource to add a relevant measurement, and clicks an appropriate button. In reaction to this event, the GUI requests the Monitoring Hub to get all possible capabilities that can be measured for the type of a resource selected by user. The Monitoring Hub passes this request to the Knowledge component. Then, a result goes back to the GUI, which can render an appropriate user interface item that will allow the user to select which capability to measure. After having selected a capability by user, GUI issues a request to add the new measurement to the Monitoring Hub. The request contains both the resource identifier and an ontology URL of the capability. Monitoring Hub then verifies, whether the selected capability can be measured with the selected resource. Such a validation is needed, because in certain circumstances it might be unfeasible, e.g. not every transport proxy is able to measure a given capability of a resource. To check this, Monitoring Hub sends a verification request to the transport proxy (the JMX Transport Proxy in the example from Figure~\ref{fig:comm_add_measurement}). After a successful verification, the Monitoring Hub initializes a scheduled job that will poll for the capability values - the measurement is successfully created and started.

\begin{figure}[ht]
\centering
\includegraphics[width=0.9\textwidth]{comm_add_measurement}
\caption{Communication diagram - adding of new measurement}
\label{fig:comm_add_measurement}
\end{figure}

Publishing a new capability value is definitely the most frequently used data flow in the whole system. Figure~\ref{fig:comm_new_cap_value} depicts this process. In contrast to the previous communication diagrams, in this case the Monitoring Hub component is the initiator of this operation. The logic of this component manages scheduled jobs responsible for polling a new capability value, and then pushing it to the listeners, the previously registered GUI components. In the first step, Monitoring Hub calls an appropriate (the one associated with resource in question) transport proxy, which is JMX Transport Proxy in this case. JMX Transport Proxy maps a generic \texttt{getCapabilityValue} request into a specific JMX query, sends the query to the monitored JVM and returns a result to the Hub. The Monitoring Hub uses this result to issue a notification dispatched through all the registered listeners. The GUI component, receives an event and updates a visualization graph, which is being watched by user.

\begin{figure}[ht]
\centering
\includegraphics[width=0.7\textwidth]{comm_new_cap_value}
\caption{Communication diagram - new capability value}
\label{fig:comm_new_cap_value}
\end{figure}


\subsection{Communication protocol}
\label{subsec:arch_comm_protocol}

The Communication protocol between components must allow both types of interactions: local and remote. By a local interaction I mean the one, where all components constitute a single process and share the same memory space. The system must use a remote interaction, when one or more components works as a separate process to other, which enforces the usage of network stack for messages exchange. To make it possible, all communications will be performed using predefined, plain interfaces, which are not aware of the underlying communication type. Such an approach decouples a communication schema from a networking protocol being employed.

As a general rule for all message exchanges between components, transfer object (also known as value object) design pattern will be used~\cite{0131422464}. This allows for decoupling the content of message of any complexity from the serialization mechanisms used. The following tables list all transfer objects used in the system. Additionally the reader may find a short description of each object;s rationale behind each member. 

% Add vertical spacing
\renewcommand*\arraystretch{1.2}

\begin{table}[ht] % ======================== CapabilityValue =====================================
\begin{tabular}{| m{1,5cm} | m{3,2cm} | m{8,5cm} |}
\hline 
\cellcolor[gray]{0.9} Field Type & \cellcolor[gray]{0.9} Field Name & \cellcolor[gray]{0.9} Details \\
\hline 
Number & \texttt{numberValue} & Numeric value of capability (optional, must be set if \texttt{ValueType} is number) \\
Number[] & \texttt{arrayValue} & Vector value of capability (optional, must be set if \texttt{ValueType} is vector) \\
ValueType & \texttt{valueType} & Type of capability value - either numeric or vector\\
Date & \texttt{gatherTimestamp} & Timestamp in UTF, when capability value have been gathered \\
String & \texttt{metricsId} & Id of measurement to which this capability value belongs \\
\hline 
\end{tabular}
\caption{List of members of \texttt{CapabilityValue} Transfer Object}
\label{tab:TO_CapValue}
\end{table} % ======================== CapabilityValue =====================================

Table~\ref{tab:TO_CapValue} contains a list of members of the \texttt{CapabilityValue} transfer object. This object is used to notify listeners (GUI) about new capability values, and acts mostly as a container for value, with additional metadata. The most significant additional property that each \texttt{CapabilityValue} has is the \texttt{gatherTimestamp} which points to the exact moment of time, when this capability value has been gathered. Using this property, the system can use \texttt{CapabilityValue} transfer objects, without worrying about the implication of processing time on a measurements presentation.

Members of the \texttt{MeasurementDefinition} message can be found in Table~\ref{tab:TO_MeasurementDef}. This object is used by the GUI component to define a measurement that should be created.

\begin{table}[ht] % ======================== MeasurementDefinition =====================================
\begin{tabular}{| m{1,5cm} | m{3,2cm} | m{8,5cm} |}
\hline 
\cellcolor[gray]{0.9} Field Type & \cellcolor[gray]{0.9} Field Name & \cellcolor[gray]{0.9} Details \\
\hline 
String & \texttt{resourceUri} & URI of resource that is covered by this measurement \\
String & \texttt{capabilityUri} & URI of capability that is covered by this measurement \\
long & \texttt{updateInterval} & Interval in milliseconds defining how frequently the value of measurement will be polled \\
String & \texttt{id} & Identifier of this measurement \\ 
\hline 
\end{tabular}
\caption{List of members of \texttt{MeasurementDefinition} transfer object}
\label{tab:TO_MeasurementDef}
\end{table} % ======================== MeasurementDefinition =====================================

The following tables:~\ref{tab:TO_Resource} and \ref{tab:TO_ResourceEvent} contain a definition of transfer objects needed for the resource management. Resource transfer object contains a complete description of resource managed by the system, additionally the Monitoring Hub uses \texttt{ResourceEvent} to notify all listeners (GUI mostly) about resource's life cycle events. 

\begin{table}[ht] % ======================== Resource =====================================
\begin{tabular}{| m{1,5cm} | m{3,2cm} | m{8,5cm} | }
\hline 
\cellcolor[gray]{0.9} Field Type & \cellcolor[gray]{0.9} Field Name & \cellcolor[gray]{0.9} Details \\
\hline
String & \texttt{typeUri} & URI of resource\rq{}s type according to currently used ontology \\
String & \texttt{uri} & URI of resource in current resources tree hierarchy \\
Map & \texttt{properties} & Static properties of resource (e.g. OS version) \\
\hline 
\end{tabular}
\caption{List of members of \texttt{Resource} transfer object}
\label{tab:TO_Resource}
\end{table} % ======================== Resource =====================================

\begin{table}[ht] % ======================== ResourceEvent =====================================
\begin{tabular}{| m{1,5cm} | m{3,2cm} | m{8,5cm} | }
\hline 
\cellcolor[gray]{0.9} Field Type & \cellcolor[gray]{0.9} Field Name & \cellcolor[gray]{0.9} Details \\
\hline
Type & \texttt{eventType} & Enumeration that defines whether resources in this event have been added or removed \\
List & \texttt{resources} & Collections of resources covered by this event \\
\hline 
\end{tabular}
\caption{List of members of \texttt{ResourceEvent} transfer object}
\label{tab:TO_ResourceEvent}
\end{table} % ======================== ResourceEvent =====================================
% Remove vertical spacing
\renewcommand*\arraystretch{1}


%---------------------------------------------------------------------------
% Monitoring Hub component.
%
%---------------------------------------------------------------------------
\section{Monitoring Hub}
\label{sec:arch_monitoring_hub}

Monitoring Hub is the core component of the system. If using layered model to analyze application, Monitoring Hub should be treated as logic layer - placed between the presentation layer (GUI) and data access layer (transport proxies). It provides services to GUI and it uses Transport Proxy implementations and Knowledge component. Its main responsibilities include resources management (registering new resources, discovery dependent resources), measurements management (creation, pausing, resuming and termination) and creation of scheduled jobs that pools for capability values and pushes new values to registered listeners.

When user wants to start monitoring new resource, He or She must choose which Monitoring Hub to use. This decision, creates direct association between the chosen hub, the resource to monitor and all its child resources discovered during registration. This association defines that Monitoring Hub associated during registration and discovery process will process all further request calls related to given resource.

To make high-level components loosely coupled, Monitoring Hub is not aware of details about other components. To be able to provide its services it uses transport proxies and knowledge components, but it interacts with them only using commonly known interfaces. Additionally it is not aware about the existence of GUI component at all. It just provide services by implementing given interface, and it uses also common listener interfaces to notify about a variety of events.


\subsection{Decomposition of Monitoring Hub}

\begin{figure}[ht]
\centering
\includegraphics[width=0.9\textwidth]{decomposition_mon_hub}
\caption{Communication diagram - adding of new resource}
\label{fig:decomposition_mon_hub}
\end{figure}

Further decomposition of Monitoring Hub can be found in Figure~\ref{fig:decomposition_mon_hub}. This module is composed of following low level components:

\begin{itemize}

\item {\bf CoreServiceFacadeImpl}~~~~~~~~~~~~~~~~~~~~~~~~~~~~~~~~~~~~~~~~~~~~~~~~~~~~~~~~\linebreak
CoreServiceFacadeImpl is a facade that allows access to all Monitoring Hub functionalities from a single interface (CoreServiceFacade, shared with GUI and Monitoring Hub Application high level components). This wrapper eases remote access by exposing and using one interface with any remoting middle ware is easier than multiple ones. Having multiple interfaces in most cases would require multiple socket connections which can be a source of additional problems. We should tend to use as small amount connections as possible, because more connections system uses, than it becomes more and more prone to network configuration issues (e.g. firewalls). Additionally, using single facade improves code style, as with facade, only one interface has to be visible for all other components that are its clients.

\item {\bf CoreRemoteServiceImpl}~~~~~~~~~~~~~~~~~~~~~~~~~~~~~~~~~~~~~~~~~~~~~~~~~~~~~~~~\linebreak
CoreRemoteServiceImpl is a service responsible for processing requests related to remote management of Monitoring Hub. It has two main responsibilities: it allows registering remote listening interfaces and it dispatches local events (new resource, new capability value etc.) to remote listeners in aggregated manner. Distributed dispatch of events requires a bit different approach than a local one (local one, means dispatch inside of single JVM process). First of all, in most cases remote interface that will receive notifications with different signature, to allow handling exceptions related to networking. Additionally, to reduce the amount of remote calls and thus improve efficiency, CoreRemoteServiceImpl aggregates events into batches and notifies remote listeners using a generated package of them. Such an approach does not pollute measuring results, because each measurement value has associated gather timestamp (see~\ref{subsec:arch_comm_protocol}), initialized by component that grabs given value, at the exact moment, of measurement. Additionally, as events related to resources addition/removal don\rq{}t require such a strict time association, those events can be aggregated without any issues.

\item {\bf CoreResourcesServiceImpl}~~~~~~~~~~~~~~~~~~~~~~~~~~~~~~~~~~~~~~~~~~~~~~~~~~~~~~~~\linebreak
CoreResourcesServiceImpl is responsible for resources management: registering new ones, discovery of their children, returning all registered and discovered ones. It is also used to get more details about a resource, like capabilities that given resource may have.

\item {\bf CoreMeasurementServiceImpl}~~~~~~~~~~~~~~~~~~~~~~~~~~~~~~~~~~~~~~~~~~~~~~~~~~~~~~~~\linebreak
The CoreMeasurementServiceImpl can be used to create, pause, stop or terminate a measurement. Additionally it gathers current values of capabilities. It is used by the PollJob component for that purpose. It implements the MeasurementListener interface to be able to receive notifications about new capability values polled by the PollJob. 

\item {\bf TransportProxiesManagerImpl}~~~~~~~~~~~~~~~~~~~~~~~~~~~~~~~~~~~~~~~~~~~~~~~~~~~~~~~~\linebreak
The TransportProxiesManagerImpl component manages registered transport proxies. It is used by other components to get all transport proxies or to lookup a transport proxy that can be used to manage given resource.

\item {\bf CapabilityValuePollManagerImpl}~~~~~~~~~~~~~~~~~~~~~~~~~~~~~~~~~~~~~~~~~~~~~~~~~~~~~~~~\linebreak
CapabilityValuePollManagerImpl is responsible for scheduling polling jobs that are needed to run measurement.

\item {\bf PollJob}~~~~~~~~~~~~~~~~~~~~~~~~~~~~~~~~~~~~~~~~~~~~~~~~~~~~~~~~\linebreak
PollJob gets triggered with a configured interval. It simply polls for current capability value and push it to listeners specified during a creation stage.

\end{itemize}

\pagebreak

\subsection{Most important data flows}

This section contains a description of most salient data flows inside Monitoring Hub component. It covers actions of adding new resources, adding new measurements and dispatching new capability value notification.

Figure~\ref{fig:comm_mh_add_res} depicts communication diagram of adding new resources. In this scenario, external GUI component initiates action - registerResource request is the first step send by GUI to CoreServiceFacadeImpl. Facade simply delegates this call to CoreResourcesServiceImpl, which will perform actual registration. In the next step, CoreResourcesServiceImpl tries to find proxy capable to communicate with given resource by calling TransportProxiesManagerImpl. What should be noticed here, Transport Proxy is an external component to Monitoring Hub. After successfully obtaining transport proxy, CoreResourcesServiceImpl performs call to register given resource. If this registration succeeds, the CoreResourcesServiceImpl uses external Knowledge component, to get all types of child resources. Using this list, service requests TransportProxy to discover all children of the registered resource. After successful discovery, notification containing resources is being dispatched to all listeners. This can be done either directly, when given Monitoring Hub is embedded into GUI or remotely, using CoreRemoteServiceImpl.

\begin{figure}[ht]
\centering
\includegraphics[width=0.9\textwidth]{comm_mh_add_res}
\caption{Monitoring Hub Communication diagram - adding new resource}
\label{fig:comm_mh_add_res}
\end{figure}

Communication diagram covering action of adding new measurements can be found in Figure~\ref{fig:comm_mh_add_measurement}. In this scenario, again GUI component is initiator. GUI component calls a getResourceCapabilities method, thus sends a first request to CoreServiceFacadeImpl. It will use those capabilities to show the user UI component, which will let user choose which capability should be measured. Facade delegates query to CoreResourcesServiceImpl, which passes it to external Knowledge component. Resulting list of capability URIs is being passed all way back, to the GUI component. Next, user selects, which capability should be measured. On selection event, GUI requests CoreServiceFacadeImpl to create measurement using given definition (see Table~\ref{tab:TO_MeasurementDef}). Request is delegated to CoreMeasurementServiceImpl which is responsible for the creation of measurement. Measurement service requests CapabilityValuePollManagerImpl to create a new instance of PollJob class. Measurement service, after creating polling job, generates measurement id, stores it internally with measurement definition, and returns it to requester. This identifier is then passed back to GUI, which will use it in any future call referring newly created measurement. Additionally it will be used to reference all incoming CapabilityValue objects.

\begin{figure}[ht]
\centering
\includegraphics[width=0.8\textwidth]{comm_mh_add_measurement}
\caption{Monitoring Hub Communication diagram - adding new measurement}
\label{fig:comm_mh_add_measurement}
\end{figure}

Last data flow covered in this section describes gathering and publishing capability values. This time, CapabilityValuePollManagerImpl initiates the action. Its internal scheduler triggers previously created PollJob which contains measurement definition (URI's of resource and capability). Using those identifiers, PollJob calls CoreMeasurementServiceImpl to getCapabilityValue. Measurement service first looks up a transport proxy using TransportProxiesManagerImpl and then, using external TransportProxy, gets the capability value. In a subsequent step, gathered capability value is being pushed to all listeners, either directly (to listeners registered at CoreMeasurementServiceImpl) or using CoreRemoteServiceImpl to all remote listeners. What should be noticed here is that CoreRemoteServiceImpl notifies remote listeners in asynchronous and aggregated manner.

\begin{figure}[ht]
\centering
\includegraphics[width=1\textwidth]{comm_mh_new_cap_val}
\caption{Monitoring Hub Communication diagram - new capability value notification}
\label{fig:comm_new_cap_val}
\end{figure}
\pagebreak

%---------------------------------------------------------------------------
% Monitoring Hub application component.
%
%---------------------------------------------------------------------------
\section{Monitoring Hub Application}
\label{sec:arch_monitoring_hub_application}

Monitoring Hub Application is a component which has only one purpose that is crucial for using the system in a distributed manner - it allows using the Monitoring Hub remotely. A deeper analysis of the Monitoring Hub Application component will not be covered, because of its rather narrow functionality. It contains a code responsible for starting a process (a class with a \texttt{main} method), initializing the components of the Monitoring Hub and resolving dependencies between the components.

Additionally, in order to provide its distributed services, the Monitoring Hub Application module is responsible for the initialization of a remoting context. This involves a creation of all remote services and export of the Monitoring Hub functionality to remote clients using a ready to use RMI middleware.


\section{Knowledge implementation}

Implementation of the Knowledge component is relatively simple and relies mostly on the ontology management library. For that purpose, the SemSimMon uses two projects - Jena\footnote{\url{http://jena.sourceforge.net/index.html}} and Pellet\footnote{\url{http://clarkparsia.com/pellet}}. 

The Knowledge component exposes functionalities to Monitoring Hub using straightforward interface, which listing can be found in Figure~\ref{fig:iknowledge_java}.

\begin{figure}[ht]
  \centering
  \begin{Verbatim}[commandchars=\\\{\},numbers=left,firstnumber=1,stepnumber=1]
\PY{k+kn}{package} \PY{n}{pl}\PY{o}{.}\PY{n+na}{edu}\PY{o}{.}\PY{n+na}{agh}\PY{o}{.}\PY{n+na}{semsimmon}\PY{o}{.}\PY{n+na}{common}\PY{o}{.}\PY{n+na}{api}\PY{o}{.}\PY{n+na}{knowledge}\PY{o}{;}

\PY{k+kn}{import} \PY{n+nn}{java.util.List}\PY{o}{;}

\PY{k+kd}{public} \PY{k+kd}{interface} \PY{n+nc}{IKnowledge} \PY{o}{\PYZob{}}

  \PY{n}{String} \PY{n+nf}{getOntologyURI}\PY{o}{(}\PY{o}{)}\PY{o}{;}

  \PY{n}{List}\PY{o}{<}\PY{n}{String}\PY{o}{>} \PY{n}{getChildrenResourceTypes}\PY{o}{(}\PY{n}{String} \PY{n}{type}\PY{o}{)}\PY{o}{;}

  \PY{n}{List}\PY{o}{<}\PY{n}{String}\PY{o}{>} \PY{n}{getCapabilitiesOfResourceType}\PY{o}{(}\PY{n}{String} \PY{n}{type}\PY{o}{)}\PY{o}{;}

\PY{o}{\PYZcb{}}
\end{Verbatim}

  \caption{Listing of IKnowledge.java}
  \label{fig:iknowledge_java}
\end{figure} 

\section{Transport Proxy Implementation}

\subsection{JMX Transport Proxy Implementation}

JMX Transport Proxy is build around two concepts: discovery agent and capability probe. Implementation consist following discovery agent classes:
\begin{itemize} 
  \item{\bf{GCDiscoveryAgent}} - discovers garbage collectors within single running JVM instance boundaries
  \item{\bf{ThreadsDiscoveryAgent}} - discovers all threads running in single JVM
  \item{\bf{CPUDiscoveryAgent}} - discovers CPUs on given computing node
  \item{\bf{JvmDiscoveryAgent}} - discovers specific JVM running on node
  \item{\bf{OsDiscoveryAgent}} - discovers operating system that runs on given node
  \item{\bf{GenericDiscoveryAgent}} - used to \lq{}discover\rq{} components that exists but any of their properties can be extracted
\end{itemize} 

Additionally, JMX Transpor proxy uses following probes to fetch capability values: 
\begin{itemize} 
  \item{\bf{GarbageCollectionsProbe}} - gets values of Garbage Collector related capabilities (CollectionCountCapability and CollectionTimeCapability)
  \item{\bf{MemoryProbe}} - gathers values memory related capabilities: total, free and used. It is used in conjunction with both physical and virtual memory resources
  \item{\bf{HeapProbe}} - measures heap usage  
  \item{\bf{ThreadTimingProbe}} - monitors threads timings: ThreadCPUTimeCapability and ThreadUserTimeCapability capabilities
  \item{\bf{ThreadSynchronizationDetailsProbe}} - monitors capabilities related to threads synchronization: ThreadBlockedCountCapability, ThreadBlockedTimeCapability, ThreadWaitedCountCapability and ThreadWaitedTimeCapability
  \item{\bf{JmxQueryCapabilityProbe}} - generic probe that can measure capabilities using JMX query given by initialization parameter
\end{itemize} 

\subsection{OCM-G Transport Proxy Implementation}

OCM-G Transport Proxy implementation uses similar approach to JMX one. In this case, are also employed two concepts - probes (works in same way as in JMX Transport Proxy) and resource agents. OCM-G proxy consists of resource agents instead of discovery agents, because OCM-G have means to actually manage resources that are monitored by this system, and those means are used in SemSimMon. Thus, resource agents have two purposes - discovery resources and manage them.

There are following resource agents implemented:

\begin{itemize} 
  \item{\bf{AppsResourceAgent}} - manages applications monitored by given MainSM
  \item{\bf{ClustersResourceAgent}} - manages clusters within given application
  \item{\bf{NodeResourceAgent}} - manages nodes within given cluster
  \item{\bf{ProcessFunctionsResourceAgent}} - manages function resources
  \item{\bf{ThreadResourceAgent}} - manages threads
  \item{\bf{CpuResourceAgent}} - manages processors
  \item{\bf{NetIfaceResourceAgent}} - manages network interfaces
  \item{\bf{OSResourceAgent}} - manages operating systems
  \item{\bf{PhysicalMemoryRA}} - manages physical memory
  \item{\bf{ProcessResourceAgent}} - manages processes
  \item{\bf{StorageResourceAgent}} - manages storage devices
  \item{\bf{VirtualMemoryRA}} - manages virtual memory
  \item{\bf{BasicHardwareResourceAgent}} - it\rq{}s type of generic hardware resource agent that can simply discover hardware devices that can\rq{}t be distinguished
\end{itemize} 
  
OCM-G Transport Proxy uses following probes:

\begin{itemize} 
  \item{\bf{LoadAvgProbe}} - monitors node\rq{}s load average 
  \item{\bf{ThreadsCP}} - monitors capability probes related to threads
  \item{\bf{TotalCpuTimeCapabilityProbe}} - measures CPU total time
  \item{\bf{FunctionProbe}} - monitors capabilities related to functions (TotalCallsTimeCapability and TotalCallsCountCapability)
\end{itemize} 

\subsection{TransportProxy interface}
Although Transport Proxies are components where most of actual magic occurs, it\rq{}s implementation is relatively simple. To use functionalities provided by this component, Monitoring Hub uses only one single interface, which contents can be seen in Figure~\ref{fig:transport_proxy}.

\begin{figure}[ht]
  \centering
  \begin{Verbatim}[commandchars=\\\{\}]
\PY{k+kn}{package} \PY{n}{pl}\PY{o}{.}\PY{n+na}{edu}\PY{o}{.}\PY{n+na}{agh}\PY{o}{.}\PY{n+na}{semsimmon}\PY{o}{.}\PY{n+na}{common}\PY{o}{.}\PY{n+na}{api}\PY{o}{.}\PY{n+na}{transport}\PY{o}{;}

\PY{k+kn}{import} \PY{n+nn}{pl.edu.agh.semsimmon.common.api.resource.IResourceDiscoveryListener}\PY{o}{;}
\PY{k+kn}{import} \PY{n+nn}{pl.edu.agh.semsimmon.common.vo.core.measurement.CapabilityValue}\PY{o}{;}
\PY{k+kn}{import} \PY{n+nn}{pl.edu.agh.semsimmon.common.vo.core.resource.Resource}\PY{o}{;}

\PY{k+kn}{import} \PY{n+nn}{java.util.List}\PY{o}{;}
\PY{k+kn}{import} \PY{n+nn}{java.util.Map}\PY{o}{;}

\PY{k+kd}{public} \PY{k+kd}{interface} \PY{n+nc}{TransportProxy} \PY{o}{\PYZob{}}

  \PY{n}{CapabilityValue} \PY{n+nf}{getCapabilityValue}\PY{o}{(}\PY{n}{Resource} \PY{n}{resource}\PY{o}{,} \PY{n}{String} \PY{n}{capabilityType}\PY{o}{)}
      \PY{k+kd}{throws} \PY{n}{TransportException}\PY{o}{;}

  \PY{n}{Map}\PY{o}{<}\PY{n}{String}\PY{o}{,} \PY{n}{CapabilityValue}\PY{o}{>} \PY{n}{getCapabilityValues}\PY{o}{(}\PY{n}{Resource} \PY{n}{resource}\PY{o}{,} 
												   \PY{n}{List}\PY{o}{<}\PY{n}{String}\PY{o}{>} \PY{n}{capabilityUris}\PY{o}{)}
      \PY{k+kd}{throws} \PY{n}{TransportException}\PY{o}{;}

  \PY{k+kt}{boolean} \PY{n+nf}{hasCapability}\PY{o}{(}\PY{n}{Resource} \PY{n}{resource}\PY{o}{,} \PY{n}{String} \PY{n}{capabilityType}\PY{o}{)}
      \PY{k+kd}{throws} \PY{n}{TransportException}\PY{o}{;}

  \PY{k+kt}{void} \PY{n+nf}{addResourceDiscoveryListener}\PY{o}{(}\PY{n}{IResourceDiscoveryListener} \PY{n}{listener}\PY{o}{)}\PY{o}{;}

  \PY{k+kt}{void} \PY{n+nf}{removeTransportProxyListener}\PY{o}{(}\PY{n}{IResourceDiscoveryListener} \PY{n}{listener}\PY{o}{)}\PY{o}{;}

  \PY{k+kt}{void} \PY{n+nf}{registerResource}\PY{o}{(}\PY{n}{Resource} \PY{n}{resource}\PY{o}{)} \PY{k+kd}{throws} \PY{n}{TransportException}\PY{o}{;}

  \PY{k+kt}{void} \PY{n+nf}{unregisterResource}\PY{o}{(}\PY{n}{Resource} \PY{n}{resource}\PY{o}{)} \PY{k+kd}{throws} \PY{n}{TransportException}\PY{o}{;}

  \PY{k+kt}{boolean} \PY{n+nf}{isResourceSupported}\PY{o}{(}\PY{n}{Resource} \PY{n}{resource}\PY{o}{)}\PY{o}{;}

  \PY{k+kt}{void} \PY{n+nf}{discoverChildren}\PY{o}{(}\PY{n}{Resource} \PY{n}{resource}\PY{o}{,} \PY{n}{List}\PY{o}{<}\PY{n}{String}\PY{o}{>} \PY{n}{types}\PY{o}{)} \PY{k+kd}{throws} \PY{n}{TransportException}\PY{o}{;}

  \PY{n}{List}\PY{o}{<}\PY{n}{Resource}\PY{o}{>} \PY{n}{discoverDirectChildren}\PY{o}{(}\PY{n}{Resource} \PY{n}{resource}\PY{o}{,} \PY{n}{String} \PY{n}{type}\PY{o}{)} 
	  \PY{k+kd}{throws} \PY{n}{TransportException}\PY{o}{;}

  \PY{k+kt}{void} \PY{n+nf}{stopResource}\PY{o}{(}\PY{n}{Resource} \PY{n}{resource}\PY{o}{)} \PY{k+kd}{throws} \PY{n}{TransportException}\PY{o}{;}

  \PY{k+kt}{void} \PY{n+nf}{pauseResource}\PY{o}{(}\PY{n}{Resource} \PY{n}{resource}\PY{o}{)} \PY{k+kd}{throws} \PY{n}{TransportException}\PY{o}{;}

  \PY{k+kt}{void} \PY{n+nf}{resumeResource}\PY{o}{(}\PY{n}{Resource} \PY{n}{resource}\PY{o}{)} \PY{k+kd}{throws} \PY{n}{TransportException}\PY{o}{;}

\PY{o}{\PYZcb{}}
\end{Verbatim}

  \caption{Listing of TransportProxy.java}
  \label{fig:transport_proxy}
\end{figure} 

Additionally, to make extending application with new transport proxies even more simple, BaseTransportProxy class was introduce, to implement most of functionalities that aren\rq{}t aware of underlaying communication mechanisms. Because of that, to add new transport proxy, one must create new class that inherits from BaseTransportProxy and implement only methods that aren\rq{}t implemented in this parent.

 
%---------------------------------------------------------------------------
% Gui component.
%
%---------------------------------------------------------------------------


\section{GUI component}
\label{sec:arch_gui}

\begin{figure}[h]
  \centering
  \includegraphics[width=0.7\textwidth]{mock_main}
  \caption{Overall system decomposition}
  \label{fig:arch_overall}
\end{figure}


\begin{figure}[h]
  \centering
  \includegraphics[width=1\textwidth]{mock_resources}
  \caption{Overall system decomposition}
  \label{fig:arch_overall}
\end{figure}

\begin{figure}[h]
  \centering
  \includegraphics[width=1\textwidth]{mock_measurements}
  \caption{Overall system decomposition}
  \label{fig:arch_overall}
\end{figure}

\begin{figure}[h]
  \centering
  \includegraphics[width=1\textwidth]{mock_vis_clean}
  \caption{Overall system decomposition}
  \label{fig:arch_overall}
\end{figure}

\begin{figure}[h]
  \centering
  \includegraphics[width=1\textwidth]{mock_vis_options}
  \caption{Overall system decomposition}
  \label{fig:arch_overall}
\end{figure}





