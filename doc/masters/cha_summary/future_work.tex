%---------------------------------------------------------------------------
% Future work.
%
%---------------------------------------------------------------------------


\section{Future work}
\label{sec:ch8_future_work}
 
Although implementation of project claims to be completed, there is also field for future improvements. Suggestions on new features that could potentially improve application functionality as well as some issues that still requires some revision can be found in list below:

\begin{itemize}
 \item {\bf Add option to use SSH tunnel in connection between GUI and Monitoring Hub Application}
 
 This will work as an option when starting new Monitoring Hub. It will allow bypassing every firewall configuration - as long as user has ability to connect to target host using the SSH, He or She will be able to use the application. Additionally, it will increase security a bit.
 
\item {\bf Add ability to monitor long running tasks or 24/7 systems }  

In current release, system was designed focusing mostly on on-line monitoring, thus for example, GUI must be connected to Monitoring Hub Application all the time. It would be great feature to be able to start remote monitoring hub, configure all resources and measurements and then detach GUI and let the monitoring hub to do it\rq{}s job. After a while or on scheduled basis, user should be able to attach to given monitoring hub and review all measurement results.

\item {\bf Application state persistence }  

Another feature increasing usability would be to allow user to save current state of work. It would increase usability a lot, if user could save connections to JMX nodes and use this configuration to easily restart working with system. 

 \item {\bf Improve the way application uses Semantic Web framework} 
     
Current implementation supports OWL only partially - it uses ontology to manage relationships between monitored resources, measurements but internal storage of resources is done in plain Java.
          
\item {\bf Java instrumentation}

Adding ability to instrument Java code will increase usability of system in context of monitoring JVM a lot. At this stage of development, user can found only that there is a problem with his or hers application. To allow tracking of an actual source of the problem, user must be able to instrument code.      
     
\item {\bf Better connections management}

Currently, application provides the user only basic management of external connections to monitoring hubs. User can add new connection while adding resource, but there is no view for checking all active connections and explicitly connect or disconnect from given monitoring hub. At this stage, I think that connections management is a bit too transparent for the end user, thus might be confusing. It will require changes only in GUI subsystem. This item overlaps a bit with adding ability to monitor long running tasks.

\item{\bf Better functions/methods management}

There should be view that will allow user to register functions of interest. Especially after realizing Java instrumentation. User should be able to define which parts of code should be monitored, to measure invocation count and total time. Currently, system monitors only some default MPI* functions and this configuration is defined in OCMG Transport Proxy component.

\end{itemize}