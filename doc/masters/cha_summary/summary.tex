%---------------------------------------------------------------------------
% Future work.
%
%---------------------------------------------------------------------------
\section{Summary}
\label{sec:summary}

As Section~\ref{sec:achievements} states, the implemented system meets most of the requirements posed at the beginning of this work. In spite of the above statement, it would be rather too optimistic to claim that it can be treated as something more than just alpha-version, an implementation draft that aims mostly at showing other developers, new possible directions and ideas.

I guess, that the most valuable idea I would like to share is to keep in mind the end-user all the development time. It is true, that SemSimMon and similar tools, targets a user type whose requirements are a bit stronger then from a casual user. Nevertheless, user\rq{}s comfort should always be something that application author should pay an attention.

Being aware of fact that creating such a complex application is almost impossible to be done by one person, I tried to focus on its extensibility. I must admit, that I am satisfied of the results - the application was implemented using a clear, self-explanatory architecture, with ease of extending. When another developer would like to add a new measurements back-end, all that needs to be done is to implement one, explicitly defined module from scratch (transport proxy) and adapt GUI to cover needed changes. This can be achieved only because the rest of components is not aware of a back-end implementation at all.
