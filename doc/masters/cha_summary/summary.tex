%---------------------------------------------------------------------------
% Future work.
%
%---------------------------------------------------------------------------
\section{Summary}
\label{sec:summary}

As was stated in Section~\ref{sec:achievements}, implemented system meets most of requirements imposed at the beginning of this work. In spite of above statement, it would be rather too optimistic to say that it can be treated as something more than just alpha-version, an implementation draft that aims mostly in showing other developers or even whole teams, new possible directions and ideas.

I guess, that the most important idea I\rq{}d like to share is to always keep in mind the end-user. It is true, that SemSimMon and similar, targets specific type of a user, type that we as developers may require a bit more than from casual user. Nevertheless, user\rq{}s comfort should always be something that application author should pay attention to.

Being aware of fact that creating such a complex application is almost impossible to be done by one person, I have been trying to focus on its extensibility. And I must admit, that I\rq{}m satisfied of the results - application was implemented using clear, self-explanatory architecture, with ease of extending. When another developer would like to add new measurements back-end, all that needs to be done is to implement one, explicitly defined module from scratch (transport proxy) and adapt GUI to cover needed changes. This can be achieved only because rest of components is not aware of particular back-end implementation at all.

Although implemented system is really basic, it doesn\rq{}t mean that it cannot be used. Due to its simplicity there are not any significant performance issues or other ones that prevents anyone from using it. I can say, as an author that if this work may ease life of any student or other user, I\rq{}ll be positive that this work was not pointless. 

