%---------------------------------------------------------------------------
% Wokring with semmon.
%
%---------------------------------------------------------------------------


\section{Working with application}
\label{sec:ch7_working_with}

Within this section, I will try to introduce the end-user with initial knowledge that needs to be learned in order to use application. Only real basics will be covered here, like system deliverable structure, basic installation guideline and requirements, etc.

The main and only one deliverable containing whole system is package built from GUI submodule. This component\rq{}s building script defines dependencies to all other components that needs to be obtained in order to use application. The application package comes in a form of zip archive or gzipped tarball. 

To build application, one should just issue {\bf mvn package} command, using any command line environment (bash, csh, cmd.exe) from project root - SemSimMon directory. As a result, application bundle can be found in directory SemSimMon/gui/target/gui-bin.{tar.gz,zip}.

Application bundle contains following items:
\begin{itemize}
  \item {\bf lib} directory containing all Java binaries. It includes libraries containing SemSimMon components as well as their dependencies.
  \item {\bf resources} directory containing other resources. Currently, within this directory, there is package containing Monitoring Hub Application component - {\bf mon-hub-app.tar.gz} archive.
  \item {\bf semsimmon-gui.bat} and {\bf semsimmon-gui.sh} scripts. Those scripts are to be used to launch the application	
\end{itemize}  

To install application, one needs just to extract above archive in desired directory. Additionally, at least at the moment, startup scripts do not bother about finding java executable, so user must have Java Runtime Environment binaries available from PATH system variable. To start GUI application, user needs to navigate to directory where it was installed, and simply launch appropriate script.

The mon-hub-app.tar.gz package is used internally by GUI to automatically deploy Monitoring Hub Application into desired host, so if user will remove the archive, automatic deployment will simply fail. Additionally, user may manually deploy Monitoring Hub. To do so, one must just transmit package to desired host, using SSH or any other remote access, unpack it into desired directory, and start by invoking script {\bf monhub-app.sh} or {\bf monhub-app.bat} that can be found inside. 

The bash version of script, takes to parameters as an input - the first one is action, and can be either \lq{}start\rq{} or \lq{}stop\rq{}. Start will just launch this component in a background, stop will terminate background process started previously. Additionally, when starting hub, one must define which interface will be used to listen for GUI connections. It can be given either as a IP address or interface\rq{}s DNS alias.  

Because Windows doesn\rq{}t support starting background tasks in one cmd.exe, the startup script on this platform takes just one parameter - listening interface. Values are same as on bash version.         