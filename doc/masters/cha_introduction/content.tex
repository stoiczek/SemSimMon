%---------------------------------------------------------------------------
% Chapter 1 - Introduction
%
%---------------------------------------------------------------------------


\chapter{Introduction}
\label{cha:intro}

%---------------------------------------------------------------------------

\section{Motivation}
\label{ch1:Motivation}

% @TODO think about motivation.

% Write about:
% - scale - current devs have tools which allows them to create stuff for hardware scaling from mobile phone/ J2SE up
% to distributed clusters
% - work on any soft has optimizations stage which is sehr important
% - how important measurements are
% - sth about distributed envirionments - they usage
% - how difficult it is to write sth on such an envs
% - mixing of languages - how funky it is to write something using multiple langs to suit it best to one's needs


%---------------------------------------------------------------------------

\section{Objectives}
\label{ch1:Objectives}

Main objective of this thesis is to design, implement and document system, that can be used to easily monitor
applications with wide range of scale. From single process applications running on one PC machine, through those
working on medium scaled clusters, up to highly distributed solutions, that use multiple computing nodes and grid
environment. Created solution should be easy to use by variety of users - students, lecturers, software engineers or
scientists. This last objective was the source of project's code name - SemSimMon, Semantic Simple Monitor. 

To extend application's life time, it should be released as open source project, and meet all criteria for such a
software: open standards, open protocols, open source dependencies. Additionally, what is probably most important,
development process should come in pair with extensive documentation, to shorten learning curve for new or additional
developers.

Another aspect of having widely used application is portability - user should be able to work with application on
majority of currently available platforms, which includes all flavors of *nixs (Unix, Linux, Mac OS X\textregistered)
and Microsoft\textregistered~Windows\textregistered. 


\section{Document structure}
\label{ch1:docStructure}

Reader of this Thesis can find its content divided into \ref{cha:summary} chapters. 

Chapter~\ref{cha:background} contains detailed description of problem domain. Covers background technologies and review
of software
with similar purpose. One may find there results of initial analysis that made rest of this work possible.

Chapters ~\ref{cha:requirements}~to~\ref{cha:deployment} focuses solely on development process of created application.
All steps of application's initial life cycle have been described - starting from gathering requirements of potential
users and formalizing them into use cases (chapter~\ref{cha:requirements}). As next step, chapter~\ref{cha:sys_arch}
I'd like to take reader through design stage. This chapter contains system analysis, top to bottom. - from
highest point of view (high level functional decomposition), down to deep insight into way each high level component
should work. 

Implementation process was described in Chapter~\ref{cha:implementation}), where we can find detailed description of
developing tools used and short overview on system from programming point of view.

Description of initial application's life cycle steps ends on Chapter~\ref{cha:deployment}, covering tests and
deployment of ready solution.

Additionally, in last chapter (Chapter~\ref{cha:summary}), author tries to summarize achieved results. Within this
part of thesis dear reader may find also short description of ideas on next steps that could be taken to let application
evolve.




