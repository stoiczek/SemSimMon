%---------------------------------------------------------------------------
% System Requirements.
%
%---------------------------------------------------------------------------
\section{System Requirements}
\label{sec:SystemRequirements}

This section contains requirements that SemSimMon, as a product must meet. Requirements listing and description is divided into two categories, namely functional and non-functional requirements. Main purpose of functional requirements analysis is to key out roughly what user can do with application, to describe its functionality. On the other hand, nonfunctional requirements add constraints to functional ones. Nonfunctional requirements analysis defines conditions under which application must operate, specify properties the software must exhibit\cite{Windle:SoftReq}.

Requirement levels described in the following Requirements specifications using terms such as ``must'', ``must not'', ``should/shall'' and ``should/shall not'' are to be read with reference to RFC 2119\cite{rfc:2119}.

\subsection{Functional Requirements}
\label{subsec:FunctionalRequirements}

Application provides to the end user following functionalities: 

\begin{itemize}

\item {\bf Connect to already running JVM process using JMX extension.} 
Application should allow user to connect to application that has been previously started. User should be able to connect to both remote and local Virtual Machine process. Additionally, to allow monitoring of environment that user has poor connectivity to, or to bypass firewall restrictions, user should be able to start component responsible for remote measuring. This component shall aggregate results from one or more measurements and send them to component controlled directly by user in packaged manner.

\item {\bf Aggregate monitored JVM processes into applications and clusters.}
User should be able to manage existing connections to processes being monitored in easy fashion, thus should be able to group them into logical units like clusters (group of nodes) or applications (collection of nodes or clusters that runs same program)

\item {\bf Connect to already running OCM-G MainSM monitor.}
User must be able to analyze performance of applications being monitored by OCM-G system. To allow this, application must provide facility for connecting to MainSM monitor, list already registered applications and attach to selected ones. Requirements related to environments to which user has limited access (in terms of network quality) applies here. Additionally, user should be able to work with both resource types (based on JMX and OCM-G monitoring beck-ends) in same time, with seamless integration (user shouldn't perform any action to switch context of work).

\item {\bf Extract and list resources of given resource.}
Application should be able to perform deep discovery mechanisms of monitored resources and visualize them.

\item {\bf Control measuring capabilities of given resource.}
Application must provide facility to start, stop, pause and resume monitoring of given resource's capability. This includes also definition of value gathering interval. Additionally user should be able to configure and update parameters of measurement, like polling frequency. 

\item {\bf List values of given resource's capabilities at current point of time.}
Per user request, application should provide value of given capability (or multiple capabilities at once) at given moment of time.

\item {\bf List all capability values from given, started measurement.}
After starting measurement user should be able to view all its values since start. Each value should be displayed with time stamp pointing to exact moment when measurement was taken. Additionally, view that will show those values must be able to self-refresh on-line in case of pending measurements.

\item {\bf Visualize measurement results.}
User must be able to select measurement and plot results on graph. User should be able to choose from more than one graph type. Each graph should be able to render results from multiple measurements. Plots should be updated on-line with new values of pending measurements.

\item {\bf List properties of given resource.}
User should be able view all static meta-data of given, selected, resource. By static meta-data I mean attributes of resource that doesn't change in time, like: host name, OS version, JVM version, etc.

\item {\bf Change ontology}
User should be able to change ontology that system internally uses for building resource tree or arranges measurements.

\end{itemize}

\subsection{Nonfunctional Requirements}
\label{subsec:NonFunctionalRequirements}

SemSimMon shall meet following nonfunctional requirements:

\begin{itemize}

\item {\bf Reliability.}
It's the most important nonfunctional requirement. Application cannot have any critical bugs that can cause it crash. Application failure, cannot lead to data loss. 

\item {\bf Scalability.}
User should be able to work with wide variety of environments - from single process running on single host, up to highly distributed application running on several clusters containing vast amount of computing units.

\item {\bf Interoperability.}
Application should operate on most currently available platforms: Linux, Microsoft\textregistered~Windows\textregistered or Mac OS X\textregistered.

\item {\bf Usability.}
User interface should be as much user-friendly as it's possible. User shouldn't lose time on learning how application actually works. Each function should be just in the place, where casual user will expect it to be.

\item {\bf Driven by Open Source.}
To allow extend of application's life cycle, its development should be based on open source resources.

\item {\bf Well documented.}
Extensive documentation is crucial to allow other developers contributing to the project.

\end{itemize}
