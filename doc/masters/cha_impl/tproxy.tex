\pagebreak
\section{Transport Proxy Implementation}
\subsection{TransportProxy interface}
Although Transport Proxies are components where most of actual magic occurs, it\rq{}s implementation is relatively simple. To use functionalities provided by this component, Monitoring Hub uses only one single interface, which contents can be seen in Figure~\ref{fig:transport_proxy}.

\begin{figure}[ht]
  \centering
  \begin{Verbatim}[commandchars=\\\{\},numbers=left,firstnumber=1,stepnumber=1]
\PY{k+kn}{package} \PY{n}{pl}\PY{o}{.}\PY{n+na}{edu}\PY{o}{.}\PY{n+na}{agh}\PY{o}{.}\PY{n+na}{semsimmon}\PY{o}{.}\PY{n+na}{common}\PY{o}{.}\PY{n+na}{api}\PY{o}{.}\PY{n+na}{transport}\PY{o}{;}

\PY{k+kn}{import} \PY{n+nn}{pl.edu.agh.semsimmon.common.api.resource.IResourceDiscoveryListener}\PY{o}{;}
\PY{k+kn}{import} \PY{n+nn}{pl.edu.agh.semsimmon.common.vo.core.measurement.CapabilityValue}\PY{o}{;}
\PY{k+kn}{import} \PY{n+nn}{pl.edu.agh.semsimmon.common.vo.core.resource.Resource}\PY{o}{;}

\PY{k+kn}{import} \PY{n+nn}{java.util.List}\PY{o}{;}
\PY{k+kn}{import} \PY{n+nn}{java.util.Map}\PY{o}{;}

\PY{k+kd}{public} \PY{k+kd}{interface} \PY{n+nc}{TransportProxy} \PY{o}{\PYZob{}}

  \PY{n}{CapabilityValue} \PY{n+nf}{getCapabilityValue}\PY{o}{(}\PY{n}{Resource} \PY{n}{resource}\PY{o}{,} \PY{n}{String} \PY{n}{capabilityType}\PY{o}{)}
      \PY{k+kd}{throws} \PY{n}{TransportException}\PY{o}{;}

  \PY{n}{Map}\PY{o}{<}\PY{n}{String}\PY{o}{,} \PY{n}{CapabilityValue}\PY{o}{>} \PY{n}{getCapabilityValues}\PY{o}{(}\PY{n}{Resource} \PY{n}{resource}\PY{o}{,} 
												   \PY{n}{List}\PY{o}{<}\PY{n}{String}\PY{o}{>} \PY{n}{capabilityUris}\PY{o}{)}
      \PY{k+kd}{throws} \PY{n}{TransportException}\PY{o}{;}

  \PY{k+kt}{boolean} \PY{n+nf}{hasCapability}\PY{o}{(}\PY{n}{Resource} \PY{n}{resource}\PY{o}{,} \PY{n}{String} \PY{n}{capabilityType}\PY{o}{)}
      \PY{k+kd}{throws} \PY{n}{TransportException}\PY{o}{;}

  \PY{k+kt}{void} \PY{n+nf}{addResourceDiscoveryListener}\PY{o}{(}\PY{n}{IResourceDiscoveryListener} \PY{n}{listener}\PY{o}{)}\PY{o}{;}

  \PY{k+kt}{void} \PY{n+nf}{removeTransportProxyListener}\PY{o}{(}\PY{n}{IResourceDiscoveryListener} \PY{n}{listener}\PY{o}{)}\PY{o}{;}

  \PY{k+kt}{void} \PY{n+nf}{registerResource}\PY{o}{(}\PY{n}{Resource} \PY{n}{resource}\PY{o}{)} \PY{k+kd}{throws} \PY{n}{TransportException}\PY{o}{;}

  \PY{k+kt}{void} \PY{n+nf}{unregisterResource}\PY{o}{(}\PY{n}{Resource} \PY{n}{resource}\PY{o}{)} \PY{k+kd}{throws} \PY{n}{TransportException}\PY{o}{;}

  \PY{k+kt}{boolean} \PY{n+nf}{isResourceSupported}\PY{o}{(}\PY{n}{Resource} \PY{n}{resource}\PY{o}{)}\PY{o}{;}

  \PY{k+kt}{void} \PY{n+nf}{discoverChildren}\PY{o}{(}\PY{n}{Resource} \PY{n}{resource}\PY{o}{,} \PY{n}{List}\PY{o}{<}\PY{n}{String}\PY{o}{>} \PY{n}{types}\PY{o}{)} \PY{k+kd}{throws} \PY{n}{TransportException}\PY{o}{;}

  \PY{n}{List}\PY{o}{<}\PY{n}{Resource}\PY{o}{>} \PY{n}{discoverDirectChildren}\PY{o}{(}\PY{n}{Resource} \PY{n}{resource}\PY{o}{,} \PY{n}{String} \PY{n}{type}\PY{o}{)} 
	  \PY{k+kd}{throws} \PY{n}{TransportException}\PY{o}{;}

  \PY{k+kt}{void} \PY{n+nf}{stopResource}\PY{o}{(}\PY{n}{Resource} \PY{n}{resource}\PY{o}{)} \PY{k+kd}{throws} \PY{n}{TransportException}\PY{o}{;}

  \PY{k+kt}{void} \PY{n+nf}{pauseResource}\PY{o}{(}\PY{n}{Resource} \PY{n}{resource}\PY{o}{)} \PY{k+kd}{throws} \PY{n}{TransportException}\PY{o}{;}

  \PY{k+kt}{void} \PY{n+nf}{resumeResource}\PY{o}{(}\PY{n}{Resource} \PY{n}{resource}\PY{o}{)} \PY{k+kd}{throws} \PY{n}{TransportException}\PY{o}{;}

\PY{o}{\PYZcb{}}
\end{Verbatim}

  \caption{Listing of TransportProxy.java}
  \label{fig:transport_proxy}
\end{figure} 

Additionally, to make extending application with new transport proxies even more simple, BaseTransportProxy class was introduce, to implement most of functionalities that aren\rq{}t aware of underlaying communication mechanisms. Because of that, to add new transport proxy, one must create new class that inherits from BaseTransportProxy and implement only methods that aren\rq{}t implemented in this parent.

\subsection{JMX Transport Proxy Implementation}

JMX Transport Proxy is build around two concepts: discovery agent and capability probe. Implementation consist following discovery agent classes:
\begin{itemize} 
  \item{\bf{GCDiscoveryAgent}} - discovers garbage collectors within single running JVM instance boundaries
  \item{\bf{ThreadsDiscoveryAgent}} - discovers all threads running in single JVM
  \item{\bf{CPUDiscoveryAgent}} - discovers CPUs on given computing node
  \item{\bf{JvmDiscoveryAgent}} - discovers specific JVM running on node
  \item{\bf{OsDiscoveryAgent}} - discovers operating system that runs on given node
  \item{\bf{GenericDiscoveryAgent}} - used to \lq{}discover\rq{} components that exists but any of their properties can be extracted
\end{itemize} 

Additionally, JMX Transpor proxy uses following probes to fetch capability values: 
\begin{itemize} 
  \item{\bf{GarbageCollectionsProbe}} - gets values of Garbage Collector related capabilities (CollectionCountCapability and CollectionTimeCapability)
  \item{\bf{MemoryProbe}} - gathers values memory related capabilities: total, free and used. It is used in conjunction with both physical and virtual memory resources
  \item{\bf{HeapProbe}} - measures heap usage  
  \item{\bf{ThreadTimingProbe}} - monitors threads timings: ThreadCPUTimeCapability and ThreadUserTimeCapability capabilities
  \item{\bf{ThreadSynchronizationDetailsProbe}} - monitors capabilities related to threads synchronization: ThreadBlockedCountCapability, ThreadBlockedTimeCapability, ThreadWaitedCountCapability and ThreadWaitedTimeCapability
  \item{\bf{JmxQueryCapabilityProbe}} - generic probe that can measure capabilities using JMX query given by initialization parameter
\end{itemize} 

\subsection{OCM-G Transport Proxy Implementation}

OCM-G Transport Proxy implementation uses similar approach to JMX one. In this case, are also employed two concepts - probes (works in same way as in JMX Transport Proxy) and resource agents. OCM-G proxy consists of resource agents instead of discovery agents, because OCM-G have means to actually manage resources that are monitored by this system, and those means are used in SemSimMon. Thus, resource agents have two purposes - discovery resources and manage them.

There are following resource agents implemented:

\begin{itemize} 
  \item{\bf{AppsResourceAgent}} - manages applications monitored by given MainSM
  \item{\bf{ClustersResourceAgent}} - manages clusters within given application
  \item{\bf{NodeResourceAgent}} - manages nodes within given cluster
  \item{\bf{ProcessFunctionsResourceAgent}} - manages function resources
  \item{\bf{ThreadResourceAgent}} - manages threads
  \item{\bf{CpuResourceAgent}} - manages processors
  \item{\bf{NetIfaceResourceAgent}} - manages network interfaces
  \item{\bf{OSResourceAgent}} - manages operating systems
  \item{\bf{PhysicalMemoryRA}} - manages physical memory
  \item{\bf{ProcessResourceAgent}} - manages processes
  \item{\bf{StorageResourceAgent}} - manages storage devices
  \item{\bf{VirtualMemoryRA}} - manages virtual memory
  \item{\bf{BasicHardwareResourceAgent}} - it\rq{}s type of generic hardware resource agent that can simply discover hardware devices that can\rq{}t be distinguished
\end{itemize} 
  
OCM-G Transport Proxy uses following probes:

\begin{itemize} 
  \item{\bf{LoadAvgProbe}} - monitors node\rq{}s load average 
  \item{\bf{ThreadsCP}} - monitors capability probes related to threads
  \item{\bf{TotalCpuTimeCapabilityProbe}} - measures CPU total time
  \item{\bf{FunctionProbe}} - monitors capabilities related to functions (TotalCallsTimeCapability and TotalCallsCountCapability)
\end{itemize} 


