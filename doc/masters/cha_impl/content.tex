 
%---------------------------------------------------------------------------
% Content of Chapter 6 - Implementation
%
%---------------------------------------------------------------------------


\chapter{Implementation details}
\label{cha:implementation}

Within this chapter dear may find rough implementation details of SemSimMon application. I don't intent to cover
source code in detail, instead I'd like to describe most important interfaces, technologies  that were
employed during implementation and will try to bring some light on source code structure. The main goal I want to
achieve with it is to lower the learning curve, needed to continue working on the project by other developers.

Each an every component of SemSimMon system was implemented using Java programming language. Decision to use this
particular technology is quite obvious, after even rough analysis of requirements that system must met. Java is
actually the most mature, cross platform programming language, so during development there is no need to worry about
platform specific issues, as it's part of Java Virtual Machine vendor responsibility. Additionally, as one of major
requirements is to be able to monitor processes running on JVM, using JMX protocol. The second communication
requirement - interactions with OMIS compliant tools, can be easily achieved using Java bindings for OCM-G
platform\footnote{\url{http://grid.cyfronet.pl/ocmg/japi.html}}.  

Although building mechanism isn't directly part of implementation, understanding this process is crucial for someone
who would like to work with project. SemSimMon uses Apache Maven\footnote{\url{http://maven.apache.org/}} as build
manager. Maven recently became the de facto standard in building Java based projects. Features like dependencies
management, build configuration instead of scripting actions needed to build project and extensive usage of plug-in
architecture. Project configuration uses maven's aggregation mechanism to ease dependencies and configuration
management. Project structure contains following maven modules:

 
\begin{itemize}
 \item pl.edu.agh.semsimmon:semsimmon - root module
 \item pl.edu.agh.semsimmon:gui - GUI component
 \item pl.edu.agh.semsimmon:mon-hub - Monitoring Hub component
 \item pl.edu.agh.semsimmon:mon-hub-app - Monitoring Hub Application component 
 \item pl.edu.agh.semsimmon:commons - component containing Transfer Objects and interfaces shared among all other
components
 \item pl.edu.agh.semsimmon:knowledge - Knowledge component
 \item pl.edu.agh.semsimmon:transports - parent module for all transport components
 \item pl.edu.agh.semsimmon.transports:jmx - JMX Transport component
 \item pl.edu.agh.semsimmon.transports:ocmg - OCM-G Transport component
\end{itemize}


Using Java as main implementation technology has another significant implication - wide usage and uncounted amount of
open source libraries that can shorten development time.



\section{Common frameworks}

There are two areas of implementation that are shared among all components, namely logging and testing. As a logging
framework, SemSimMon uses combined slf4j\footnote{\url{http://www.slf4j.org}} and
log4j\footnote{\url{http://logging.apache.org/log4j}} libraries. Apache Log4j is probably most widely used logging
framework for Java and is known for it's low performance impact and extensive configuration options. slf4j stands for
Simple Logging Facade for Java, it's main purpose is to simplify logging interface even more, and decouple code from
actual logging framework. This is quite important, as logging invocations tends to spread over code, which makes any
change in logging facility quite a challenging task. With slf4j changing logging framework requires only change in
configuration code.

As a testing framework, SemSimMon uses 2 libraries: TestNG\footnote{\url{http://testng.org/doc/index.html}} and
mockito\footnote{\url{http://mockito.org/}}. TestNG provides test runners and means to implement test cases. On the
other hand, mockito is excellent library for implementing unit tests - it provides facility for creating easy to use
mock objects, which are essential in Test Driven Development.

\section{GUI Implementation}

\subsection{Frameworks}

The most important framework used by GUI is Apache Pivot\footnote{\url{http://pivot.apache.org/}}. As authors say,
Apache Pivot is an open-source platform for building installable Internet applications (IIAs). It is quite new Apache
project, which brings completely new approach to GUI building in Java. The biggest feature of Pivot is totally
decoupled interface description from programming language. The framework uses XML markup from bxml files to build
interfaces. Example of such a file, can be found in Figure~\ref{fig:bxml_example}
 
\begin{figure}[ht]
  \centering
    
\begin{Verbatim}[commandchars=\\\{\},frame=single,framerule=0.2pt]                                                     
\PY{n+nt}{<Window} \PY{n+na}{title=}\PY{l+s}{"Hello BXML!"} \PY{n+na}{maximized=}\PY{l+s}{"true"}                      
    \PY{n+na}{xmlns:bxml=}\PY{l+s}{"http://pivot.apache.org/bxml"}                                                      
    \PY{n+na}{xmlns=}\PY{l+s}{"org.apache.pivot.wtk"}\PY{n+nt}{>}                                                       
    \PY{n+nt}{<Label} \PY{n+na}{text=}\PY{l+s}{"Hello BXML!"}                                                           
        \PY{n+na}{styles=}\PY{l+s}{"\PYZob{}font:'Arial bold 24', color:'#ff0000',}                                     
\PY{l+s}{            horizontalAlignment:'center', verticalAlignment:'center'\PYZcb{}"}\PY{n+nt}{/>}                    
\PY{n+nt}{</Window>}                                                                                                    
\end{Verbatim} 
  \caption{Example of BXML syntax}
  \label{fig:bxml_example}
\end{figure}

Another great feature of Pivot is data binding. Pivot provides own collections which triggers updates of UI state, on
change of collection content. Additionally, it provides means to bind UI items with controller class members. Those
features simplify realization of MVC design pattern a lot. All views in GUI component are implemented using BXML and are
binded to controllers. What is also worth to mention, Pivot allows to easily internationalize application, and
SemSimMon uses it's i18n features - all texts in labels and other UI widgets are internationalized. The default,
English localization has been provided.

Second most important framework in use by GUI is well known Spring
Framework\footnote{\url{http://www.springsource.org}}. Spring is used to integrate all components of this module,
to associate controllers with views and with model. For that purpose spring Inversion of Control Container is
employed\footnote{\url{http://static.springsource.org/spring/docs/3.0.x/reference/beans.html}}

\subsection{Controller infrastructure}

Controllers are most important components in MVC, and thus needs special attention. There are few core mechanisms in
SemSimMon that needs to be understood in order to work with GUI module. 

All controllers inherits from common super class, namely BaseController. This class contains pivot component (view)
that given controller manages, resource (bxml file) that must be used to read and instantiate view and i18n resources. 
It's responsibility is to create component from resource, bind all widgets and button actions. 

