%---------------------------------------------------------------------------
% Content of Chapter 6 - Implementation
%
%---------------------------------------------------------------------------
\chapter{Implementation details}
\label{cha:implementation}

\chabstract{
The reader may find rough details of a SemSimMon application implementation within this chapter. I do not intend to cover the source code in detail, instead I would like to describe most critical interfaces, technologies that were employed during implementation and will try to bring some light on a source code structure. The main purpose of this chapter is to lower the learning curve, for other developers who will further evolve the project.
}


\section{Introduction}

Each and every component of SemSimMon system was implemented using Java programming language. The decision to use this technology is quite obvious, after even a brief analysis of requirements that the system must meet. Java is the most mature, cross platform programming language, so during development there is no need to worry about platform-specific issues, as it is part of Java Virtual Machine vendor responsibility. Additionally, one of utmost requirements is to monitor processes running on the JVM, using the JMX protocol, which forces the usage of Java. The second communication requirement, interactions with OMIS compliant tools, can be easily achieved using Java bindings for OCM-G platform\footnote{\url{http://grid.cyfronet.pl/ocmg/japi.html}}.

Although build process is not directly part of implementation, understanding this process is crucial for someone who would like to work with the project. The SemSimMon uses the Apache Maven\footnote{\url{http://maven.apache.org/}} as a build manager. Maven recently became a de facto standard in helping managing Java based projects. The Project configuration uses Maven's aggregation facility to ease the dependencies and configuration management. The project structure contains following maven modules: 

\begin{itemize}
\item \texttt{pl.edu.agh.semsimmon:semsimmon} - root module
\item \texttt{pl.edu.agh.semsimmon:gui} - GUI component
\item \texttt{pl.edu.agh.semsimmon:mon-hub} - Monitoring Hub component
\item \texttt{pl.edu.agh.semsimmon:mon-hub-app} - Monitoring Hub Application component 
\item \texttt{pl.edu.agh.semsimmon:commons} - component containing Transfer Objects and interfaces shared among all other components
\item \texttt{pl.edu.agh.semsimmon:knowledge} - Knowledge component
\item \texttt{pl.edu.agh.semsimmon:transports} - parent module for all transport components
\item \texttt{pl.edu.agh.semsimmon.transports:jmx} - JMX Transport component
\item \texttt{pl.edu.agh.semsimmon.transports:ocmg} - OCM-G Transport component
\end{itemize}

\section{Common frameworks}

Using Java as the main implementation technology has another significant implication - wide usage and uncounted amount of open source libraries that can simplify development.

There are two areas of implementation that all components share, namely logging and testing. As a logging framework, the SemSimMon uses combined slf4j\footnote{\url{http://www.slf4j.org}} and log4j\footnote{\url{http://logging.apache.org/log4j}} libraries. Apache Log4j is probably the most widely used logging framework for Java and has low performance impact and extensive configuration options. Slf4j stands for Simple Logging Facade for Java. Its main purpose is to simplify logging interface even more and decouple code from the logging framework used. This is quite noteworthy, as logging invocations tends to spread over code, which makes any change in a core logging facility quite a challenging task. With slf4j changing a logging framework requires only change in configuration code.

As a testing framework, SemSimMon uses 2 libraries: TestNG\footnote{\url{http://testng.org/doc/index.html}} and mockito\footnote{\url{http://mockito.org/}}. TestNG provides test runners and means to implement test cases. On the other hand, mockito is an excellent library for implementing unit tests - it provides a facility for creating easy to use mock objects, which are essential in Test Driven Development.

 
%---------------------------------------------------------------------------
% Gui component.
%
%---------------------------------------------------------------------------


\section{GUI component}
\label{sec:arch_gui}

\begin{figure}[h]
  \centering
  \includegraphics[width=0.7\textwidth]{mock_main}
  \caption{Overall system decomposition}
  \label{fig:arch_overall}
\end{figure}


\begin{figure}[h]
  \centering
  \includegraphics[width=1\textwidth]{mock_resources}
  \caption{Overall system decomposition}
  \label{fig:arch_overall}
\end{figure}

\begin{figure}[h]
  \centering
  \includegraphics[width=1\textwidth]{mock_measurements}
  \caption{Overall system decomposition}
  \label{fig:arch_overall}
\end{figure}

\begin{figure}[h]
  \centering
  \includegraphics[width=1\textwidth]{mock_vis_clean}
  \caption{Overall system decomposition}
  \label{fig:arch_overall}
\end{figure}

\begin{figure}[h]
  \centering
  \includegraphics[width=1\textwidth]{mock_vis_options}
  \caption{Overall system decomposition}
  \label{fig:arch_overall}
\end{figure}


\section{Knowledge implementation}

Implementation of the Knowledge component is relatively simple and relies mostly on the ontology management library. For that purpose, the SemSimMon uses two projects - Jena\footnote{\url{http://jena.sourceforge.net/index.html}} and Pellet\footnote{\url{http://clarkparsia.com/pellet}}. 

The Knowledge component exposes functionalities to Monitoring Hub using straightforward interface, which listing can be found in Figure~\ref{fig:iknowledge_java}.

\begin{figure}[ht]
  \centering
  \begin{Verbatim}[commandchars=\\\{\},numbers=left,firstnumber=1,stepnumber=1]
\PY{k+kn}{package} \PY{n}{pl}\PY{o}{.}\PY{n+na}{edu}\PY{o}{.}\PY{n+na}{agh}\PY{o}{.}\PY{n+na}{semsimmon}\PY{o}{.}\PY{n+na}{common}\PY{o}{.}\PY{n+na}{api}\PY{o}{.}\PY{n+na}{knowledge}\PY{o}{;}

\PY{k+kn}{import} \PY{n+nn}{java.util.List}\PY{o}{;}

\PY{k+kd}{public} \PY{k+kd}{interface} \PY{n+nc}{IKnowledge} \PY{o}{\PYZob{}}

  \PY{n}{String} \PY{n+nf}{getOntologyURI}\PY{o}{(}\PY{o}{)}\PY{o}{;}

  \PY{n}{List}\PY{o}{<}\PY{n}{String}\PY{o}{>} \PY{n}{getChildrenResourceTypes}\PY{o}{(}\PY{n}{String} \PY{n}{type}\PY{o}{)}\PY{o}{;}

  \PY{n}{List}\PY{o}{<}\PY{n}{String}\PY{o}{>} \PY{n}{getCapabilitiesOfResourceType}\PY{o}{(}\PY{n}{String} \PY{n}{type}\PY{o}{)}\PY{o}{;}

\PY{o}{\PYZcb{}}
\end{Verbatim}

  \caption{Listing of IKnowledge.java}
  \label{fig:iknowledge_java}
\end{figure} 

\section{Transport Proxy Implementation}

\subsection{JMX Transport Proxy Implementation}

JMX Transport Proxy is build around two concepts: discovery agent and capability probe. Implementation consist following discovery agent classes:
\begin{itemize} 
  \item{\bf{GCDiscoveryAgent}} - discovers garbage collectors within single running JVM instance boundaries
  \item{\bf{ThreadsDiscoveryAgent}} - discovers all threads running in single JVM
  \item{\bf{CPUDiscoveryAgent}} - discovers CPUs on given computing node
  \item{\bf{JvmDiscoveryAgent}} - discovers specific JVM running on node
  \item{\bf{OsDiscoveryAgent}} - discovers operating system that runs on given node
  \item{\bf{GenericDiscoveryAgent}} - used to \lq{}discover\rq{} components that exists but any of their properties can be extracted
\end{itemize} 

Additionally, JMX Transpor proxy uses following probes to fetch capability values: 
\begin{itemize} 
  \item{\bf{GarbageCollectionsProbe}} - gets values of Garbage Collector related capabilities (CollectionCountCapability and CollectionTimeCapability)
  \item{\bf{MemoryProbe}} - gathers values memory related capabilities: total, free and used. It is used in conjunction with both physical and virtual memory resources
  \item{\bf{HeapProbe}} - measures heap usage  
  \item{\bf{ThreadTimingProbe}} - monitors threads timings: ThreadCPUTimeCapability and ThreadUserTimeCapability capabilities
  \item{\bf{ThreadSynchronizationDetailsProbe}} - monitors capabilities related to threads synchronization: ThreadBlockedCountCapability, ThreadBlockedTimeCapability, ThreadWaitedCountCapability and ThreadWaitedTimeCapability
  \item{\bf{JmxQueryCapabilityProbe}} - generic probe that can measure capabilities using JMX query given by initialization parameter
\end{itemize} 

\subsection{OCM-G Transport Proxy Implementation}

OCM-G Transport Proxy implementation uses similar approach to JMX one. In this case, are also employed two concepts - probes (works in same way as in JMX Transport Proxy) and resource agents. OCM-G proxy consists of resource agents instead of discovery agents, because OCM-G have means to actually manage resources that are monitored by this system, and those means are used in SemSimMon. Thus, resource agents have two purposes - discovery resources and manage them.

There are following resource agents implemented:

\begin{itemize} 
  \item{\bf{AppsResourceAgent}} - manages applications monitored by given MainSM
  \item{\bf{ClustersResourceAgent}} - manages clusters within given application
  \item{\bf{NodeResourceAgent}} - manages nodes within given cluster
  \item{\bf{ProcessFunctionsResourceAgent}} - manages function resources
  \item{\bf{ThreadResourceAgent}} - manages threads
  \item{\bf{CpuResourceAgent}} - manages processors
  \item{\bf{NetIfaceResourceAgent}} - manages network interfaces
  \item{\bf{OSResourceAgent}} - manages operating systems
  \item{\bf{PhysicalMemoryRA}} - manages physical memory
  \item{\bf{ProcessResourceAgent}} - manages processes
  \item{\bf{StorageResourceAgent}} - manages storage devices
  \item{\bf{VirtualMemoryRA}} - manages virtual memory
  \item{\bf{BasicHardwareResourceAgent}} - it\rq{}s type of generic hardware resource agent that can simply discover hardware devices that can\rq{}t be distinguished
\end{itemize} 
  
OCM-G Transport Proxy uses following probes:

\begin{itemize} 
  \item{\bf{LoadAvgProbe}} - monitors node\rq{}s load average 
  \item{\bf{ThreadsCP}} - monitors capability probes related to threads
  \item{\bf{TotalCpuTimeCapabilityProbe}} - measures CPU total time
  \item{\bf{FunctionProbe}} - monitors capabilities related to functions (TotalCallsTimeCapability and TotalCallsCountCapability)
\end{itemize} 

\subsection{TransportProxy interface}
Although Transport Proxies are components where most of actual magic occurs, it\rq{}s implementation is relatively simple. To use functionalities provided by this component, Monitoring Hub uses only one single interface, which contents can be seen in Figure~\ref{fig:transport_proxy}.

\begin{figure}[ht]
  \centering
  \begin{Verbatim}[commandchars=\\\{\}]
\PY{k+kn}{package} \PY{n}{pl}\PY{o}{.}\PY{n+na}{edu}\PY{o}{.}\PY{n+na}{agh}\PY{o}{.}\PY{n+na}{semsimmon}\PY{o}{.}\PY{n+na}{common}\PY{o}{.}\PY{n+na}{api}\PY{o}{.}\PY{n+na}{transport}\PY{o}{;}

\PY{k+kn}{import} \PY{n+nn}{pl.edu.agh.semsimmon.common.api.resource.IResourceDiscoveryListener}\PY{o}{;}
\PY{k+kn}{import} \PY{n+nn}{pl.edu.agh.semsimmon.common.vo.core.measurement.CapabilityValue}\PY{o}{;}
\PY{k+kn}{import} \PY{n+nn}{pl.edu.agh.semsimmon.common.vo.core.resource.Resource}\PY{o}{;}

\PY{k+kn}{import} \PY{n+nn}{java.util.List}\PY{o}{;}
\PY{k+kn}{import} \PY{n+nn}{java.util.Map}\PY{o}{;}

\PY{k+kd}{public} \PY{k+kd}{interface} \PY{n+nc}{TransportProxy} \PY{o}{\PYZob{}}

  \PY{n}{CapabilityValue} \PY{n+nf}{getCapabilityValue}\PY{o}{(}\PY{n}{Resource} \PY{n}{resource}\PY{o}{,} \PY{n}{String} \PY{n}{capabilityType}\PY{o}{)}
      \PY{k+kd}{throws} \PY{n}{TransportException}\PY{o}{;}

  \PY{n}{Map}\PY{o}{<}\PY{n}{String}\PY{o}{,} \PY{n}{CapabilityValue}\PY{o}{>} \PY{n}{getCapabilityValues}\PY{o}{(}\PY{n}{Resource} \PY{n}{resource}\PY{o}{,} 
												   \PY{n}{List}\PY{o}{<}\PY{n}{String}\PY{o}{>} \PY{n}{capabilityUris}\PY{o}{)}
      \PY{k+kd}{throws} \PY{n}{TransportException}\PY{o}{;}

  \PY{k+kt}{boolean} \PY{n+nf}{hasCapability}\PY{o}{(}\PY{n}{Resource} \PY{n}{resource}\PY{o}{,} \PY{n}{String} \PY{n}{capabilityType}\PY{o}{)}
      \PY{k+kd}{throws} \PY{n}{TransportException}\PY{o}{;}

  \PY{k+kt}{void} \PY{n+nf}{addResourceDiscoveryListener}\PY{o}{(}\PY{n}{IResourceDiscoveryListener} \PY{n}{listener}\PY{o}{)}\PY{o}{;}

  \PY{k+kt}{void} \PY{n+nf}{removeTransportProxyListener}\PY{o}{(}\PY{n}{IResourceDiscoveryListener} \PY{n}{listener}\PY{o}{)}\PY{o}{;}

  \PY{k+kt}{void} \PY{n+nf}{registerResource}\PY{o}{(}\PY{n}{Resource} \PY{n}{resource}\PY{o}{)} \PY{k+kd}{throws} \PY{n}{TransportException}\PY{o}{;}

  \PY{k+kt}{void} \PY{n+nf}{unregisterResource}\PY{o}{(}\PY{n}{Resource} \PY{n}{resource}\PY{o}{)} \PY{k+kd}{throws} \PY{n}{TransportException}\PY{o}{;}

  \PY{k+kt}{boolean} \PY{n+nf}{isResourceSupported}\PY{o}{(}\PY{n}{Resource} \PY{n}{resource}\PY{o}{)}\PY{o}{;}

  \PY{k+kt}{void} \PY{n+nf}{discoverChildren}\PY{o}{(}\PY{n}{Resource} \PY{n}{resource}\PY{o}{,} \PY{n}{List}\PY{o}{<}\PY{n}{String}\PY{o}{>} \PY{n}{types}\PY{o}{)} \PY{k+kd}{throws} \PY{n}{TransportException}\PY{o}{;}

  \PY{n}{List}\PY{o}{<}\PY{n}{Resource}\PY{o}{>} \PY{n}{discoverDirectChildren}\PY{o}{(}\PY{n}{Resource} \PY{n}{resource}\PY{o}{,} \PY{n}{String} \PY{n}{type}\PY{o}{)} 
	  \PY{k+kd}{throws} \PY{n}{TransportException}\PY{o}{;}

  \PY{k+kt}{void} \PY{n+nf}{stopResource}\PY{o}{(}\PY{n}{Resource} \PY{n}{resource}\PY{o}{)} \PY{k+kd}{throws} \PY{n}{TransportException}\PY{o}{;}

  \PY{k+kt}{void} \PY{n+nf}{pauseResource}\PY{o}{(}\PY{n}{Resource} \PY{n}{resource}\PY{o}{)} \PY{k+kd}{throws} \PY{n}{TransportException}\PY{o}{;}

  \PY{k+kt}{void} \PY{n+nf}{resumeResource}\PY{o}{(}\PY{n}{Resource} \PY{n}{resource}\PY{o}{)} \PY{k+kd}{throws} \PY{n}{TransportException}\PY{o}{;}

\PY{o}{\PYZcb{}}
\end{Verbatim}

  \caption{Listing of TransportProxy.java}
  \label{fig:transport_proxy}
\end{figure} 

Additionally, to make extending application with new transport proxies even more simple, BaseTransportProxy class was introduce, to implement most of functionalities that aren\rq{}t aware of underlaying communication mechanisms. Because of that, to add new transport proxy, one must create new class that inherits from BaseTransportProxy and implement only methods that aren\rq{}t implemented in this parent.


